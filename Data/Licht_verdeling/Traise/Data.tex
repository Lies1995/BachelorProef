\documentclass{article}
%\usepackage[utf8]{inputenc}
\usepackage{amsmath,amsfonts,amsthm} % Math packages

\title{Data voor de temperatuursverhoging simulatie}
\author{Lies Deceuninck en Hannelore Verhoeven}
\date{October 2015}

\usepackage{natbib}
\usepackage{graphicx}

\begin{document}

\maketitle
In onderstaande tabel staan de data die we gebruikt hebben als input in de code.\\

\begin{tabular}{c||c|c}
 & $474 nm$ & $560 nm$\\ \hline
$u_a$ [$m^{-1}$] & $51.1$ & $107.9$ \\
$u_s$ [$m^{-1}$] & $12733$ & $9266$ \\
$g$ & $0.88$ & $0.88$ \\
$\rho$ [$\frac{kg}{m^3}$] & $1040$ & $1040$ \\
$c$ [$\frac{J}{kgK}$] & $3650$ & $3650$ \\
$k$ [$\frac{W}{mK}$] & $0.530$ & $0.530$ \\
$W_L$ [$m$] & $0.0000045$ & $0.0001$ \\
\end{tabular}
\vspace{1cm}\\
Hieronder is het resultaat voor de tijdsconstante tau.\\

\begin{tabular}{c||c|c}
& $\tau_{SM}$ [$10^{-2} ms$]& $\tau_{MM} [ms]$ \\ \hline
$474 nm$ & $1.334524$ & $6.572070$ \\
$560 nm$ & $1.334527$ & $6.579400$ \\
\end{tabular}
\\
Hierin is geen rekening gehouden met beduidende cijfers. Wel is er afgerond volgens de gebruikelijke regels. 

\end{document}