\documentclass[paper=a4, fontsize=11pt]{scrartcl} % A4 paper and 11pt font size

\usepackage[T1]{fontenc} % Use 8-bit encoding that has 256 glyphs
\usepackage{fourier} 
\usepackage{graphicx}
\graphicspath{{gfx/}}
\usepackage[english]{babel} % English language/hyphenation
\usepackage{amsmath,amsfonts,amsthm} % Math packages
\usepackage{xcolor}
\usepackage{url}
\usepackage{array}
\usepackage{marginnote}
\usepackage[top=2cm, bottom=2cm, outer=5cm, inner=2cm, heightrounded, marginparwidth=3.5cm, marginparsep=0.5cm]{geometry}

\usepackage{sectsty} % Allows customizing section commands
\allsectionsfont{\centering \normalfont\scshape} % Make all sections centered, the default font and small caps

\usepackage{fancyhdr} % Custom headers and footers
\pagestyle{fancyplain} % Makes all pages in the document conform to the custom headers and footers
\fancyhead[R]{Lies Deceuninck, Hannelore Verhoeven} % No page header - if you want one, create it in the same way as the footers below
\fancyfoot[L]{Bachelor Thesis(1) 2015} % Empty left footerœ
\fancyfoot[C]{KuLeuven} % Empty center footer
\fancyfoot[R]{\thepage} % Page numbering for right footer
\renewcommand{\headrulewidth}{0pt} % Remove header underlines
\renewcommand{\footrulewidth}{0pt} % Remove footer underlines
\setlength{\headheight}{13.6pt} % Customize the height of the header

%\numberwithin{equation}{section} % Number equations within sections (i.e. 1.1, 1.2, 2.1, 2.2 instead of 1, 2, 3, 4)
%\numberwithin{figure}{section} % Number figures within sections (i.e. 1.1, 1.2, 2.1, 2.2 instead of 1, 2, 3, 4)
%\numberwithin{table}{section} % Number tables within sections (i.e. 1.1, 1.2, 2.1, 2.2 instead of 1, 2, 3, 4)

\setlength\parindent{0pt} % Removes all indentation from paragraphs - comment this line for an assignment with lots of text
%----------------------------------------------------------------------------------------
%	TITLE SECTION
%----------------------------------------------------------------------------------------

\newcommand{\horrule}[1]{\rule{\linewidth}{#1}} % Create horizontal rule command with 1 argument of height

\title{	
\normalfont \normalsize 
\textsc{Kuleuven, Department of Physics and Astronomy} \\ [25pt] % Your university, school and/or department name(s)
\horrule{0.5pt} \\[0.2cm] % Thin top horizontal rule
\huge Bachelor Thesis\\ % The assignment title
\horrule{2pt} \\[0.3cm] % Thick bottom horizontal rule
}

\author{Lies Deceuninck and Hannelore Verhoeven} % Your name

\date{\normalsize\today} % Today's date or a custom date

\begin{document}

\maketitle % Print the title
\section{Bekomen Data uit Trise plot} % (fold)
\label{sec:data}

% section section_name (end)

In tabel \ref{tab:const} staan de data die we gebruikt hebben als input in de code.\\
\begin{table}[h!]
\caption{Gebruikte constanten}
\centering
\begin{tabular}{c||c|c}
 & $474 nm$ & $560 nm$\\ \hline
$\mu_a$ [$m^{-1}$] & $51.1$ & $107.9$ \\
$\mu_s$ [$m^{-1}$] & $12733$ & $9266$ \\
$g$ & $0.88$ & $0.88$ \\
$\rho$ [$\frac{kg}{m^3}$] & $1040$ & $1040$ \\
$c$ [$\frac{J}{kg\cdot K}$] & $3650$ & $3650$ \\
$k$ [$\frac{W}{m\cdot K}$] & $0.530$ & $0.530$ \\
$w_L^{SM}$ [$m$] & $0.0000045$ & $0.0000045$  \\
$w_L^{MM}$ [$m$] & $0.0001$ & $0.0001$  \\
\end{tabular}
\label{tab:const}
\end{table}

In tabel \ref{tab:tau} staan de bekomen resultaten voor de tijdsconstante tau (zie vergelijking \ref{eq:tau}) \\

\begin{equation}
\tau = \frac{\rho c}{k(2.4)^2}\left[\frac{r^2_0}{1+\left(\frac{r_0 \pi}{4.8z_0}\right)^2}\right]
\end{equation}
\begin{subequations}

met constanten
\begin{align}
        z_0 &= \frac{2}{\mu_a + (1-g)\mu_s} = \frac{2}{\mu_a+\mu_s'}\\
		r_0 &= w_L.
\end{align}
 waarbij $z_0$ en $r_0$ respectievelijk de axiale en radiale limieten zijn van de temperatuursverhoging.
\label{eq:tau}
\end{subequations}
\marginnote{ Dat $z_0$ en $r_0$ de limieten zijn voor de temperatuursverhoging betekend concreet $\delta T(t,z,r)=0 \forall r\geq r_0 of z\geq z_0$}
\begin{table}
\centering
\caption{Bekomen tijdsconstanten $\tau$ voor elk protocol}
\begin{tabular}{c||c|c}
& $\tau_{SM}$ [$10^{-2} ms$]& $\tau_{MM} [ms]$ \\ \hline
$474 nm$ & $2.517970$ & $12.401371$ \\
$560 nm$ & $2.517975$ & $12.414704$ \\
\end{tabular}
\label{tab:tau}
\end{table}

Hierin is geen rekening gehouden met beduidende cijfers. Wel is er afgerond volgens de gebruikelijke regels. 
\section{Interpretaties resultaten} % (fold)
\label{sec:int-res}

\subsection{Saturatie} % (fold)
\label{sub:saturatie}
\emph{Waarom is er saturatie aan het einde van de stimulatie (zowel continue als gepulste)}
\begin{itemize}
	\item continue stimulaite:\\
	bij stimulaitie worden de moleculen geexiteerd naar hoger energie niveau door absorptie van fotonen.
	Door conductie wordt hun energie verspreid over het (oneindige) weefsel. --> energieverlies
	\begin{equation}
		\text{Je krijgt } x \text{ je geeft } \frac{-1}{2}x \text{ af netto stijging } \rightarrow  \frac{1}{2}x
	\end{equation}
	Door continue energie toevoer en energie verlies wordt uiteindelijk een evenwicht bereikt.Constante stijging van de temperatuur-->saturatie 
	Dit is uiteraard op een hoger energie nieveau (temperatuur) dan initieel 
	\item gepulste stimulatie \\
	Eenderzijds tijdens de puls krijg je na $x$ aantal tijd (ongeveer 4 keer $\tau$ ) om wille van bovenstaande redenen. Wanneer de puls stopt daalt de temperatuurstijging met dezelfde tijdsconstante. Wanneer de volgende puls start is er niet triviale bijdrage van vorige puls. Zo voelt elke puls bijdragen van de voorgaande pulsen. De temperatuursstijging is hoger dan bij de eerste puls.
	Echter, omdat een dalende exponentiele (afhankelijk de tijdsconstante) (snel) naar 0 gaat, wordt de bijdrage voor de volgende pulsen uiteindelijk triviaal. Hierdoor voelt elke puls enkel de bijdrage van $x$-aantal pulsen ervoor. Dit is een constante bijdrage waardoor de temperatuursstijging satureerd. 

\end{itemize}
% subsection saturatie (end)
\subsection{Van wat hangt de tijdsconstante $\tau$ af} % (fold)

De golflengthe $\lambda$ bepaald de absorptiecoefficient $\mu_a$ en scatteringcoefficient $\mu_s$, m.a.w $z_0$ in $\tau$. De diameter $d$ van de fiber is $2w_L=d$.
Omdat $z_0^{474nm}=0.0013 m^-1$ en $z_0^{560nm}=0.0016 m^-1$ in dezelfde grote orde zitten en de stralen $r_0$ grote orde $10^2$ verschillen zie tabel \ref{tab:const}. Door te kijken naar vergelijking \ref{eq:tau} zien we dat $\tau$ dus voornamelijk veranderd door de grote veranderingen in $r_0$, dus de diameter heeft meer invloed dan de golflengte.

% subsection van_wat_hangt_ (end)
\subsection{Temperatuursstijging voor gepulste stimulatie}
\begin{table}[]
\centering
\caption{My caption}
\label{my-label}
\begin{tabular}{l|l|l||l|l}
\hline
Irradiantie{[}mW/mm^2{]} & Puls lengte {[}ms{]} & Frequentie {[}Hz{]} & \multicolumn{2}{c}{Maximale temperatuurstijging}{[}mK{]} \\
                  &                      &                     & \lambda = 474nm     & \lambda = 560nm     \\ 
\hline
\hline
70.5              & 5                    & 5                   & 4.54                &9.44                     \\
                  &                      & 20                  & 4.62                &9.61                     \\
                  &                      & 130                 & 9.81                &20.4                     \\
380               & 30                   & 5                   & 67.2                &140                     \\
                  &                      & 10                  & 67.2                &140                     \\
                  &                      & 20                  & 68.4                &142                     \\
46                & 1                    & 40                  & 0.798               &1.66                     \\
10                & 4                    & 5                   & 0.535               &1.11                     \\
                  &                      & 10                  & 0.535               &1.11                     \\
                  &                      & 45                  & 0.642               &1.34                     \\
5                 & 20                   & 8                   & 0.777               &1.62                    
\end{tabular}
\end{table}


\end{document}