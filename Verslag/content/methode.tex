\section{Methode} % (fold)
\label{sec:methode}
   \begin{figure*}[!ht]
		\centering
		\includegraphics[width=\textwidth]{gfx/Scat.png}
		\caption{Absorptieco\"effici\"ent en gereduceerde verstrooiingsco\"effici\"ent voor de hippocampus bij ratten. De grote piek bij $560$nm komt door de aanwezigheid van hemoglobine}
	\label{fig:AbsEnScat}
\end{figure*} 
 Simulaties voor de temperatuursverandering in weefsel met de tijdsconstanten methode (zie paragraaf \ref{sec:theoretische_achtergrond}) worden uitgevoerd voor de hippocampus regio in rattenhersenen (zie figuur \ref{fig:ratbrein}). De weefsel eigenschappen zoals $\mu_a$ en $\mu_s'$\footnote{$\mu_s'=(1-g)\mu_s$} zijn golflengte afhankelijk zoals blijkt uit figuur \ref{fig:AbsEnScat}. Er wordt gefocust op protocols met golflengten $474nm$ en $560nm$ voor zowel een single-mode ($d=0.009nm$, NA=$0.12$) als multi-mode ($d=0.300nm$, NA=$0.37$) optische vezel. Er werd gekozen voor deze twee golflengten, omdat het veel gebruikte opsine ChR2 optimaal werkt bij stimulatie met een golflengte $\approx 470nm$ en er een hoge piek is bij de absorptieco\"effici\"ent voor een golflengte van $560nm$. 
Dit is ideaal om de invloed van de absorptie op de temperatuursverandering na te gaan. De gebruikte waarden voor de thermische en weefsel eigenschappen voor de hippocampus staan in tabel \ref{tab:thermEig}. De fouten op $g$, n en de thermische eigenschappen kunnen verwaarloosd worden in vergelijking met de fouten op de weefsel parameters (zie figuur \ref{fig:AbsEnScat}).
\begin{table*}[!ht]
	\caption{Thermische en weefseleigenschappen van rat hippocampus}
	\label{tab:thermEig}
	\centering

	\begin{tabular}{l|c}
	\hline
	\hline
	\textbf{Parameter} & \textbf{Waarde}\\
	\hline
	Dichtheid $\rho$	 &  $1040 \frac{kg}{m^3}$ \\
	Warmtecapaciteit $c$ & $3650 \frac{mJ}{g\cdot K}$\\
	Thermische conductiviteit $k$ & $0.530 \frac{W}{m\cdot K}$\\
		Anisotropie factor $g$ & $0.88$ \\
	Brekingsindex n & $1.4$\\
	\multirow{2}{*}{Absorptieco\"effici\"ent $\mu_a$} & $474nm$: $0.511 \pm 0.349 cm^{-1}$\\
 	&  $560nm$:$1.079^{+0.559}_{-0.525} {cm^-1}$\\
 	\multirow{2}{*}{Verstrooiings co\"effii\"ent $\mu_s$} & $474nm$: $127.33 \pm 32.67 {cm^-1}$ \\
 	&  $560nm$:$92.66^{+35.58}_{34.92}{cm^-1}$\\
	\hline
	\hline
	\end{tabular}
\end{table*} 
\subsection{Monte-Carlo simulatie} % (fold)
\label{sub:monte_carlo_simulatie}
Als eerste moet de lichtverdeling $\phi(r,z)$ bepaald worden. De lichtverdeling is een grootheid die aangeeft hoe de energie per oppervlak van de invallende fotonen zich verdeeld over het weefsel. Om meer precies te zijn, het geeft in elk punt de verandering van de energie per eenheid van oppervlakte, per eenheid van tijd. De lichtverdeling hangt af van zowel de weefsel eigenschappen (absorptieco\"efficient $\mu_a$, verstrooingsco\"fficient $\mu_s$, brekingsindex n, anisotropie factor $g$ ) als van de lichtstraal eigenschappen (golflengte $\lambda$, numerieke apertuur $NA$, diameter $d$). Om $\phi(r,z)$ te modelleren wordt een Monte-Carlo simulatie uitgevoerd. De energieverdeling wordt gezien als een \emph{random walk} van fotonen. 
\begin{figure}[!ht]
	\centering
	\includegraphics[width=8cm]{gfx/MCML.png}
	\caption{Random walk van \'e\'en foton}
	\label{fig:MCML}
\end{figure}
Het principe wordt uitgelegd aan de hand van een simpel voorbeeld; het zenden van 1 foton doorheen het weefsel (zie figuur \ref{fig:MCML}). Wanneer een foton het oppervlak van het weefsel bereikt, wordt het verstrooid en of geabsorbeerd door moleculen. Dit proces kan gemodelleerd worden als een foton met een statistisch gewicht dat een random walk uitvoert. Na elke stap verliest het een deel van zijn gewicht.  Hoeveel energie hij verliest en de geprefereede staprichting wordt random gekozen binnen een verdelingsfunctie bepaald door de weefsel parameters. Wanneer het gewicht onder een bepaalde drempel is gekomen stopt de simulatie voor dat foton. Indien dit herhaald wordt voor $1$ miljoen fotonen kan een duidelijk beeld verkregen worden van hoe de energie verdeeld wordt over het weefsel. Aan de hand van een statistisch wiskundig protocol kan vervolgens de fotonverdeling omgezet worden naar kwantitatieve gegevens voor $\phi(r,z)$. 
Een programma geschreven door \emph{Lihong Wang en Steven L. Jacques} aan  de universiteit van Texas doet dergelijke Monte-Carlo simulaties en geeft uiteindelijk lichtverdeling benaderende voor het gewenst gebied. Voor meer details wordt verwezen naar de publicatie \cite{MCML}.
\subsection{Temperatuursverandering simulatie} % (fold)
\label{sub:temperatuursverandering_simulatie}
Om de temperatuursveranderingen te modelleren volgens de benaderende oplossing besproken in paragraaf \ref{sec:theoretische_achtergrond} werd een matlab code geschreven, zie hiervoor de bijlage. 
De code bestaat uit drie scripts. Een eerste om de data file voor lichtverdeling van het Monte-Carlo simulatie programma in te lezen en de resultaten grafisch weer te geven. In deze code wordt de data voor de lichtverdeling corresponderend met een eenheids irradiantie ($\psi(r,z)$) opgeslaan als een .csv bestand. Deze kan dan gebruikt worden om in het tweede script de temperatuursverandering voor een continue stimulatie in elk punt te berekenen en grafisch weer te geven. Hiervoor wordt de oplossing van de bioheat vergelijking (vergelijking \ref{eq:solBH}) gebruikt. Een derde script berekent de temperatuursverandering voor een gepulste stimulatie.
Zoals in de vorige paragraaf reeds uitgelegd is, is de temperatuursverandering voor een gepulste stimulatie een sommatie van indiviuele korte continue stimulaties (zie figuur \ref{fig:puls4}). Welke functies er opgeteld moeten worden hangt af van de tijd. Noem $t_L$ de lengte van de puls en $F-t_L$ de tijd tussen twee individuele pulsen. De temperatuursverandering kan dan als volgt opgedeeld worden. 

\footnotesize
\begin{equation}
\Delta T(t) = \left\{\begin{array}{l l} 
\Delta T_1(t) & \text{ voor }0<t<t_L  \\
\Delta T_2(t) & \text{ voor }t_L<t<F \\ 
\Delta T_3(t) & \text{ voor }F<t<F+t_L \\
\Delta T_4(t) & \text{ voor }F+t_L<t<2F\\
\Delta T_5(t) & \text{ voor }2F<t<2F+t_L \\
\ldots
\end{array}
\label{eq:sompuls}
\end{equation}
\normalsize
Tijdens een puls is de temperatuursverandering hetzefde als bij een gewone continue stimulatie, een stijgende exponenti\"ele functie.
\begin{equation}
\Delta T_s(t) = \frac{\tau \mu_a \phi (z,r)}{\rho c} (1-e^{-\frac{t}{\tau}})
\end{equation}
Wanneer nu de stimulatie stop daalt de temperatuur exponentieel met dezelfde tijdsconstante.
\begin{equation}
	\Delta T_d(t) = T_0 e^{\frac{-t}{\tau}}
\end{equation}
Met
\begin{equation}
T_0 = \frac{\tau \mu_a \phi (z,r)}{\rho c} (1-e^{-\frac{t_L}{\tau}})
\end{equation}
De temperatuursverandering in elk stukje van de tijd wordt nu
\footnotesize
\begin{equation*} 
\begin{array}{l}
\Delta T_1(t)= \Delta T_s(t) \\
\Delta T_2(t)= \Delta T_d(t-t_L) \\
\Delta T_3(t)= \Delta T_d(t-t_L) + \Delta T_s(t-F) \\
\Delta T_4(t)= \Delta T_d(t-t_L) + \Delta T_d(t-F-t_L)\\
\Delta T_5(t)= \Delta T_d(t-t_L) + \Delta T_d(t-F-t_L) + \Delta T_s(t-2F)
\end{array}
\end{equation*}
\normalsize
Dit is het principe waarop de code voor de temperatuursverandering voor een gepulste stimulatie gebaseerd is. 

