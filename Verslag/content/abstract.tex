\Abstract{
Optogenetics is a technique which uses light to control neurons in rat brain. These neurons have been genetically modified to express opsins, which are light sensitive ion channels. In optogenetic experiments the most important interactions between light and the brain tissue, are thermal effects. Therefore in this paper the temperature increase due to laser and LED stimulation are calculated for different protocols (474nm/560nm and SM/MM) and for continuous and pulsed stimulation. For these calculations an approximate solution, which uses time constant analysis, of the bioheat equation was used. Because of the approximations, this model has limited validity, which causes it to be invalid for pulsed stimulation with a SM fiber. However it provides physical insight in the influences of tissue parameters on the temperature increase.   
}