\Abstract{
Optogenetics is a technique which uses light to control neurons in rat brain. These neurons have been genetically modified to express opsins, which are light sensitive ion channels. 
A problem with optogenetics is that interaction between light and the brain tissue causes thermal effects, which can lead to tissue damage. 
Therefore in this paper the temperature increase due to laser and LED stimulation are calculated for different protocols (474nm/560nm and SM/MM) and for continuous and pulsed stimulation. For these calculations an approximate solution of the bioheat equation was used, which uses time constant analysis. Because of the approximations, this model has limited validity, which causes it to be only valid for pulsed stimulation with a MM fiber. However it provides physical insight in the influences of tissue parameters on the temperature increase.   
}