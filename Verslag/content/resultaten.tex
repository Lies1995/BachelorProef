\section{Resultaten en Discussie} % (fold)
\label{sec:resultaten}
Simulaties voor de temperatuursverandering in weefsel met de tijdsconstanten methode worden uitgevoerd voor de hippocampus regio in het rat brein. Er wordt gefocusd op protocols voor golflengten $474nm$ en $560nm$ en voor zowel een single-mode ($d=0.009nm$, NA=$0.12$)als multi-mode ($d=0.300nm$, NA=$0.37$) optische vezel. Deze twee golflengte omdat een veel gebruikt opsin $ChR2$ optimaal werkt bij een golflengt $\approx 470nm$ en er een hoge piek is bij de absorptieco\"effici\"ent voor een golflengte vaan $560nm$. Dit is ideaal om de invloed van de absorptie op de temperatuursverandering na te gaan.  
\subsection{Energieverdeling} % (fold)
\label{sub:energieverdeling}
Het \emph{Monte-Carlo multi layer} programma geeft $\psi(r,z)$, de fluence rate corresponderend met een eenheid irradiantie $I=1\frac{W}{m^2}$. Om vervolgens de eigenlijke fluence rate $\phi(r,z)$ voor een bepaald stimulatie protocol te bekomen wordt vergelijking \ref{} gebruikt. Figuur \ref{} geeft $\phi(r,z)$ weer voor een stimulatie van laserlicht met een power van $1mW$ en voor de vier verschillende protocols waarop gefocusd wordt in dit onderzoek. 
De contourlijn op $1\frac{mW}{mm^2}$ geeft het maximale oppervlak aan waarin opsin stimulatie kan optreden, zie paragraaf \ref{sub:interactie_van_weefsel_met_elektromagnetische_straling}.  
%\onecolumn
%\begin{wrapfigure}{r}{0.01\textwidth}
%\begin{minipage}{\textwidth}
\begin{figure*}[!ht]%#TODO; eenheden +andere schaling
  	\centering
  	\includegraphics[width=\textwidth]{gfx/FR_P1mW.png}
  	\caption{Caption here}
  	\label{fig:FR}
  \end{figure*}
%\end{minipage} 
 % \end{wrapfigure} 
  %\twocolumn
Een aantal dingen die onmiddellijk opvallen bij de vergelijking van de energieverdeling voor de verschillende protocols. Wanneer de twee golflengten vergeleken worden valt op dat voor $474nm$ de energie meer radiaal verspreidt wordt, terwijl bij de $560nm$ vooral in de axiale riching verdeeld wordt. Een hoge absorptie zorgt dus voor minder verspreiding in de radiale richting en meer in axiale richting. 

De energieverdelingen voor de verschillende optische vezels (NA=$0.12$ en NA=$0.37$) verschillend vooral wat betreft de energieverspreiding. Bij een single-mode (NA=$0.12$) is de meeste energie geconcentreerd rond de oorsprong terwijl bij de multi-mode een meer homogene verdeling van de energie plaatsvindt. Dit is wat te verwachten was. Een kleine NA betekent een kleine verspreidings hoek waaronder het laserlicht aangevoerd wordt naar het weefsel. Alle energie wordt op een kleine plek geconcentreerd en dus het grootste deel van de energie wordt op dezelfde plek geabsorbeerd. Wanneer de energie echter vanaf het begin meer gespreid wordt aangevoerd (multi-mode) zal deze zich meer evenredig verspreiden over het oppervlak. De maximale energieconcentratie bij de SM is dus een factor $10^2$ groter. 

Er is een poging ondernomen om de resolutie van de afbeeldingen van $\phi(r,z)$ te verbeteren. Vooral bij de simulaties voor de SM zou dit nuttig zijn. Als invoer in het MCML-programma moet je naast de thermische en weefsel eigenschappen ook het aantal afgevuurde fotonen en het gebied specifici\"eren. Het gebied waarin de fotonen mogen wandelen is een rooster met allemaal roosterpunten. Het aantal moet dus gespecifici\"eerd worden en de roosterafstand. 
Een eerste mogelijkheid is het verhogen van het aantal afgevuurde fotonen. Dit vraagt echter veel rekenvermogen, dus gepast technologisch materiaal is hiervoor vereist.
Een tweede mogelijkheid is een zelfde aantal fotonen afvuren maar het gebied verkleinen, dus de roosterafstand verkleinen. Echter moet opgepast worden dat het gebied niet te klein is, de fotonen mogen niet in de buurt van de rand komen omdat daar geen rekening mee is gehouden in de code. Dit is een aantal keer geprobeerd maar er trad elke een probleem op met de schaling. Het gebied in een richting (axiaal of radiaal) is gelijk aan het aantal roosterpunten vermenigvuldigd met de roosterafstand. De eersta schaling was $1000$ roosterpunten in de axiale richting en $500$ in de radiale. De roosterafstand was twee keer $10^-4 cm$. De afmetingen van het gebied waarin de fluence rate gesimuleerd moet worden is: $0.1cmx0.05cm$. Echter krijgen we uit de data van het programma afmetingen van $1cmxO.5cm$. Wanneer nu de rooster afstand gehalveerd word, waren de afmetingen wel wat verwacht was, namelijk $0.5cmx0.25cm$. Het eigenaardige is vooral dat er twee keer exact dezelfde code wordt gebruikt, zowel het MCML simulatie programma als de matlab code die de data-files inleest, omzet en grafisch weergeeft. Dit is een probleem waar niet onmiddellijk een oplosing voor gevonden kon worden omdat de code voor het MCML programma niet beschikbaar is. De simulaties zijn verschillende keren herhaald met steed hetzelfde eigenaardige resultaat. Echter wordt niet uitgesloten dat het eventueel toch ergens een menselijke fout is in de inleescode. Indien betere resolutie van de fluence rate noodzakelijk is dit iets wat de moeite waard is om na te kijken. 
Verder is er ook een typfout bij de eenheden in de resulterende data file gevonden. Bij het omzetten van de fotonendistributie naar een fysische grootheid zijn er verschillende mogelijkheden. 
\begin{itemize}
	\item .Frzc, Fluence rate [$W\cdot cm^-2$]
	\item .Arz, Absorptie [$W\cdot cm^-3$]
	\item \ldots
\end{itemize}
De extensie die hier nodig is, is de .Frzc en in deze datafile staat als eenheid bij de fluence rate [$W\cdot cm^-3$]. Aangezien een andere extensie deze eenheid al heeft, is dit een verwarrende fout die uit de code gehaald moet worden. 

\subsection{Temperatuurstijging bij continue stimulatie}
De resultaten voor de fluence rate bekomen met de monte carlo simulatie werden gebruikt om de temperatuurstijging voor continue stimulatie te bepalen. Dit werd gedaan voor alle vier de protocols, telkens bij een vermogen van $1mW$. Er kunnen nu verschillende figuren gemaakt worden.
Als eerste werden er kleurenplots gemaakt die de tijdsevolutie van de temperatuursverandering weergeven. De resultaten voor \’{e}\’{e}n van de protocols, namelijk $\lambda = 560nm$ en multi-mode, staan afgebeeld in figuur \ref{fig:dT_560_MM}, die voor de andere protocols kan men vinden in de bijlage (figuur \ref{fig:dT_Overig_Bijlage}. De temperatuurstijging werd telkens geplot op drie verschillende tijdstippen. Een eerste tijdstip werd gekozen vlak nadat het laserlicht op het weefsel begint te schijnen in dit geval $t=0.001s$. Als tweede tijdstip werd telkens $t=\tau $ gekozen en als laatste tijdstip werd $t=3\tau$, dit tijdstip is de uiterste grens waarop ons model geldig is. Op dit laatste tijdstip is de temperatuur reeds gesatureerd, zodat de rechtse afbeelding telkens de finale temperatuurstijging weergeeft. Op deze afbeeldingen is duidelijk de evolutie van de temperatuursverandering zichtbaar. In het begin is de maximale temperatuurstijging lager en is de verandering minder verspreidt. Naarmate de tijd vordert wordt de maximale temperatuur groter en verspreidt de temperatuurstijging zich verder doorheen het weefsel. De fouten op deze figuren werden wel berekend, maar niet afgebeeldt omdat dat de figuren te ingewikkeld zou maken.  
% figuur
% label{fig:dT_560_MM}
\begin{figure*}[tb]
  	\centering
  	\includegraphics[width=\textwidth]{gfx/dT_560_MM.png}
  	\caption{Caption here}
  	\label{fig:dT_560_MM}
  \end{figure*} 
\\
Op figuren \ref{fig:dT_tijd_SM}  en \ref{fig:dT_tijd_MM} werd de temperatuurstijging van het hersenweefsel geplot in functie van de tijd op $z=0$ en $r=0$, telkens voor de twee verschillende golflengtes. Op de grafieken werd telkens $t=\tau$ aangeduid met een stippellijn. Aangezien de tijdsconstante voor eenzelfde vezel (SM of MM) zeer weinig van elkaar verschillen (zie later) kon er per afbeelding \’{e}\’{e}n stippellijn getekend worden. Merk op dat opnieuw de fouten niet geplot werden. De rede hiervoor is dat doordat de fout tijdsafhankelijk is, ze zeer snel, zeer groot wordt, waardoor de grafieken onduidelijk worden. Wanneer men naar de vorm van de grafieken kijkt, is te zien dat de temperatuurstijging inderdaad vrij snel zal satureren. Anderzijds is het opvallend dat de temperatuurstijging voor $560nm$ veel groter is dan voor $474nm$. Dit komt omdat de absorptie co\”effici\”ent van het weefsel een piek vertoont rond $550nm$. Rond $550nm$ zal de absorptie van fotonen door het weefsel veel groter zijn omwille van absorptie door hemoglobine. \cite{ ThermEff } Hierdoor zal de absorptie co\”effici\”ent bij $560nm$ groter zijn dan bij $474nm$ en aangezien absorptie verantwoordelijk is voor temperatuurstijging, verklaart dit de grotere temperatuursverandering bij $560nm$.
%figuur
% label{fig:dT_tijd_SM}
%label{fig:dT_tijd_MM}
\begin{figure}[tb]
  	\centering
  	\includegraphics[width=10cm]{gfx/dT_tijd_SM.png}
  	\caption{Caption here}
  	\label{fig:dT_tijd_SM}
  \end{figure} 
	
	\begin{figure}[tb]
  	\centering
  	\includegraphics[width=10cm]{gfx/dT_tijd_MM.png}
  	\caption{Caption here}
  	\label{fig:dT_tijd_MM}
  \end{figure} 
\\
Zoals daarnet al werd vermeld, is er een zeer klein verschil tussen de tijdsconstanten voor eenzelfde vezel. Dit kan men zien in tabel \ref{table:tau}. Voor verschillende golflengten is $\tau$ min of meer gelijk. De rede hiervoor is dat de waarde van de optische parameters (deze zijn afhankelijk van de golflengte) weinig invloed heeft op $\tau$. Ze heffen elkaar als het ware op. Anderzijds heeft de gebruikte vezel een zeer grote invloed op de waarde van $\tau$. De rede hiervoor is dat de tijdsconstante zeer sterk afhankelijk is van de straal van de optische vezel ($w_L$). Men kan zien dat voor de muli-mode vezel de tijdsconstanten veel groter zijn dan voor single-mode. De rede hiervoor is vrij eenvoudig intui\”tief te verklaren. Een SM vezel is smaller, waardoor alle energie zich in een klein en smal gebied bevindt. Hierdoor kan de temperatuur makkelijker en dus sneller stijgen, wat leidt tot een kleine $\tau$. Voor de MM vezel is dezelfde energie over een groter gebied verspreidt, waardoor de temperatuur trager zal stijgen. Dit geeft een grotere waarde voor de tijdsconstante. 
\begin{figure*}[h!]
    \centering
    \includegraphics[width=0.8\textwidth]{gfx/4dT.png}
    \caption{Caption here}
    \label{fig:4dT}
  \end{figure*}
% tabel met tau
\begin{table}
\centering
\caption{Bekomen tijdsconstanten $\tau$ voor elk protocol}
\begin{tabular}{c||c|c}
& $\tau_{SM}$ [$10^{-2} ms$]& $\tau_{MM} [ms]$ \\ \hline
$474 nm$ & $2.517970$ & $12.401371$ \\
$560 nm$ & $2.517975$ & $12.414704$ \\
\end{tabular}
\label{tab:tau}
\end{table}

Tot slot kunnen we de kleurenplots voor de vier protocols met elkaar vergelijken. Afbeelding \ref{fig:4dT} geeft de temperatuurstijging weer voor de verschillende protocols, telkens op een tijdstip waarop de temperatuur reeds gesatureerd is. Men kan een aantal zaken opmerken. Ten eerste komt de vorm sterk overeen met die van de afbeeldingen voor de fluence rate. Dat is logisch aangezien in een gebied met een hoge fluence rate de temperatuur sterk zal stijgen en in een gebied met een lage fluence rate, zal de temperatuur minder sterk stijgen. Ook is te zien dat de temperatuursverandering voor $560nm$ telkens groter en verder verspreid is dan voor $474nm$. Dit komt, zoals eerder al vermeld, door de hogere absorptie bij $560nm$. 

 
