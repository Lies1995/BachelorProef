\section{Inleiding} % (fold)
\label{sec:inleiding}
Neurowetenschap houdt zich bezig met de werking van het zenuwstelsel. Momenteel is er over de hele wereld veel onderzoek op de hersenen en begrijpt men elk jaar meer over de werking van het menselijk brein. Vaak gebruikte proefdieren voor neurologische proeven zijn apen, ratten, muizen, vissen,\ldots. In dit onderzoek specifici\"eren we ons op het rattenbrein.\\
Om de werking van het hersenen te begrijpen is een eerste grote en noodzakelijke stap het controleren van de activiteit van de hersencellen, \emph{neuronen}.  Neuronen zijn cellen die voordurend signalen ontvangen en versturenen door ons hele lichaam. Wanneer de activiteit kan geinhibeerd worden of een neuron kan gestimuleerd worden om signalen te sturen kan de functie van die cel achterhaald worden. Bijvoorbeeld, wanneer een neuron geinhibeerd wordt tijdens een test waarbij het proefdier moet onthouden waar een beloning ligt, en hij achteraf niet meer weet waar de beloning lag, kan worden afgeleidt dat deze cel verantwoordelijk is voor ruimtelijke orientatie. Activiteit controle van de hercencellen is dus essenti\"eel in neurologisch onderzoek en er zijn dan ook heel wat technieken voor ontwikkeld.

Een eerste mogelijkheid is elektrische stimulatie van de neuronen met electroden. Omdat de signalen tussen neuronen van elektrische aard zijn is dit een logische techniek. Een tweede mogelijkheid is het gebruik van electromagnetische straling om neuronen te activeren of inhiberen.  Hierbij worden de neuronen genitsch gemanipuleerd zodat ze lichtgevoelig worden, \emph{Optogenetica}. Helaas interageert electromagnetische straling met weefsels en kan het dus nog andere reacties induceren \footnote{zie sectie }.  %#TODO
 Een belangrijk effect is de temperatuursopwarming en dat is het onderwerp van dit onderzoek. 

\subsection{Optogenetica} % (fold)
\label{sub:optogenetica}
Eerder werd al gezegd dat de signalen tussen neuronen van elektrische aard waren. Hoe werkt dit nu precies? 
Neuronen hebben net zoals alle andere cellen in ons lichaam een plasmamembraan die de scheiding vormt tussen het intercellulair en extracellulair medium. Dit bestaat uit een dubbellaag van fosfolipieden en membraanprote\"inen, zie figuur \ref{fig:Opsins}%#TODO. 
Deze laatste functioneren als doorgangskanalen voor moleculen en zijn dan ook verantwoordlijk voor de communicatie van cellen onder elkaar. Een belangrijke soort van membraanprote\"inen zijn ionenkanalen die specifiek ionen transporteren. Door dit transport ontstaat er een potentiaal over het plasmamembraan. Een cel in rust heeft een membraanpotentiaal van ongeveer $-70mV$. 

Om een signaal te sturen moet de membraanpotentiaal omhoog. Dit kan door positieve ionen te transporteren in de cel. Dit gebeurt eerst traag tot een bepaalde treshold potentiaal bereikt wordt. %#TODO
Op dat moment openen de ionenkanalen volledig en stijgt de membraanpotentiaal exponentieel. %#TODO: afbeelding 
Dit neuron signaal wordt een  actie potentiaal genoemd. Belangrijk om op te merken is dat de actie potentiaal automatisch gegenereerd wordt eens de tresholdpotentiaal bereikt is. 
Om de celactiviteit te inhiberen kan het omgekeerde gebeuren. De membraanpotentiaal naar beneden brengen door negatieve ionen de transporteren en het bereiken van de tresholdpotentiaal bemoeilijken.   
\begin{figure}[h!]
	\centering
	\includegraphics[width=8cm]{gfx/Opsins.png}
	\caption{Caption here}
	\label{fig:Opsins}
\end{figure}
\begin{figure}[h!]
	\centering
	\includegraphics[width=8cm]{gfx/interaction.png}
	\caption{Caption here}
	\label{fig:interaction}
\end{figure}