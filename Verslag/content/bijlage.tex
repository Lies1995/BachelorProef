\section*{Bijlage} % (fold)
\label{sec:bijlage}
\subsection*{Matlabcode} % (fold)
\label{sub:matlabcode}
\subsubsection*{Lichtverdeling} % (fold)
\label{ssub:lichtverdeling}
Onderstaande matlabcode leest de .frzc files van de monte-carlo simulatie en geeft ze grafisch weer in een \emph{imagesc}. De .frzc files zijn de lichtverdelingen voor de verschillende protocols gesimuleerd met een input power $1mW$ en een gaussische lichtbundel. Er werden 1 miljoen fotonen afgevuurd.
De bestanden voor de verschillende protocols zijn alsvolgt gelabeld:
$L\_\text{\emph{golflengte}}\_N\_\text{\emph{numerieke apertuur}}$. Ze bevinden zich in de map \emph{Output}. De lichtverdeling $\phi^+$ (met $(\mu_a + \sigma) \text{ en } (\mu_s+\sigma)$) en $\phi^-$ (met $(\mu_a - \sigma) \text{ en } (\mu_s-\sigma)$) worden op dezelfde wijze gelabeld en staan in respectivelijk de mappen \emph{Output+} en \emph{Output-}. Het script schrijft de data van de lichtverdeling corresponderend met een eenheids irradiantie ($\psi(r,z)$) voor elk protocol ook als een .csv bestand naar de map \emph{unitFR}.\\
\textbf{MCML$\_$plot.m}
\begin{lstlisting}
MCML_Setup_Data
                                                    
for i = 1: length(L);
    for j=1:length(NA)
        %------DATA--------
        n_L=L(i); n_NA=NA(j);               
        protocol_name = ['L_' num2str(n_L) '_NA_' num2str(n_NA) ];
        %------Calculation--------
        for l=1:3
            %Calc. Fluence rate  [mm mm W/m^2]
            [r,z,fRate,filename]=...
                MCML_Data(n_L,n_NA,file_path,protocol_name,l); 
            %Calc. Fluence rate I=Unit peak irradiance  
            psi=fRate.*((pi*w_L(j)^2)/2).*10^3; 
            %Convert psi to Fluence rate phi for the asked Power/Irradiance
            if ne(PL,0) == 1
                IL=PL/((pi*w_L(j)^2)/2);
            end
            phi=IL.*psi;
            
            %Save all data
            dF(:,:,l)=phi; %[W/m^2]
            
            fullfilename=[fullfile(file_path,'UnitFR', filename)];
            filename_r=[fullfile(file_path,'FR_r')];
            filename_z=[fullfile(file_path,'FR_z')];
            csvwrite(fullfilename,psi); %[W/m^2]
            csvwrite(filename_r, r);%[mm]
            csvwrite(filename_z, z);%[mm]
        end
        %------Plot--------
        %Change units
        dF(:,:,:)=dF(:,:,:).*10^-3; %[mW/mm^2]
        
        fig=figure(k); 
        hold on
        %Colorgraph
        imagesc(r,z,dF(:,:,1));c=colorbar; 
        
        %Contourlines
        [C, h]=contour(r,z,dF(:,:,1),v,'color','w');
        [Cplus, hpuls]=contour(r,z,dF(:,:,2),v,'--','color','w');
        [Cminus, hminus]=contour(r,z,dF(:,:,3),v,':','color','w');
        
        title({'Fluence Rate';['$\lambda=$' num2str(n_L) ', NA=' ...
            num2str(n_NA) ', Input Irradiance=' num2str(IL) ]},...
            'interpreter', 'LaTex');
        xlabel('r [mm]', 'interpreter', 'LaTex');
        ylabel('z [mm]', 'interpreter', 'LaTex');
        c.Label.String= '$\phi(r,z)$ [$mW \cdot mm^{-2}$]';
        c.Label.Interpreter='latex';
        c.Label.FontSize=20;
        
        xlim([-0.6 0.6])
        ylim([0 1.5])
        set(gca,'ydir','rev')
        hold off
        
        %------Calculation Stimulated Volume--------
        %Collect the datapoints of the concerning contourline
        if j==1
            s=find(C(1,:)==1,2)
            [y_0,Iy_0]=max(C(2,2:s(length(s))-1));
            s=find(Cplus(1,:)==1,2)
            [y_0plus,Iy_0plus]=max(Cplus(2,2:s(length(s))-1));
             s=find(Cminus(1,:)==1,2)
            [y_0minus,Iy_0minus]=max(Cminus(2,2:s(length(s))-1));
        elseif j==2
            [y_0,Iy_0]=max(C(2,2:length(C(1,:))));
            [y_0plus,Iy_0plus]=max(Cplus(2,2:length(Cplus(1,:))));
            s=find(Cminus(1,:)==1,2)
            [y_0minus,Iy_0minus]=max(Cminus(2,2:s(length(s))-1));
        end
        %Rotate the graph 90degrees
        x=C(2,2:Iy_0);
        y=C(1,2:Iy_0);
        
        xplus=Cplus(2,2:Iy_0plus);
        yplus=Cplus(1,2:Iy_0plus);
 
        xminus=Cminus(2,2:Iy_0minus);
        yminus=Cminus(1,2:Iy_0minus);
        
        %Fit with a polynomial of degree g
        p=polyfit(x,y,g);
        fun=@(t) pi*(polyval(p,t)).^2;
        int=integral(fun,x(1),x(length(x)))
        
        pplus=polyfit(xplus,yplus,g);
        fun=@(t) pi*(polyval(pplus,t)).^2;
        intplus=integral(fun,xplus(1),xplus(length(xplus)))
        
        pminus=polyfit(xminus,yminus,g);
        fun=@(t) pi*(polyval(pminus,t)).^2;
        intminus=integral(fun,xminus(1),xminus(length(xminus)))
        
        
        %Uncomment this section if you want to check the polynomal fit
        %------Plot polynomial fit--------
%         k=k+1;
%         figure(k)
%         plot(x,y,'--','color','k')
%         hold on
%         plot(x, polyval(p,x),'color','k');
%         
%         plot(xplus,yplus,'--','color','r');
%         plot(xplus, polyval(pplus,xplus),'color','r');
%      
%         plot(xminus,yminus,'--','color','b');
%         plot(xminus, polyval(pminus,xminus),'color','b');
%       
%       hold off
        k=k+1;
    end
end

\end{lstlisting}
\textbf{MCML$\_$Setup$\_$Data.m}
\begin{lstlisting}
%------Setup--------
    close all
    % Change default text font size
        set(0,'DefaultAxesFontSize', 18)
    %File Path
        file_path=path; 
%------DATA--------
    %Wavelengths(nm)
        L = [474,560]; 
    %Numerical Aperture (dimensionless)
        NA = [12,37];  
    %Diameter (mum)
        d = [9,200];
    %1/e^2 radius [m]
        w_L= [0.0000045 0.0001];

    %Choose the stimulation, specify either the input power or the input
    %irradiace. Set the other to the default value
    %Laser Power [W] (default 0)
        PL=0.001;
    %Irradiance [W/m^2] (default 1)
        IL=1;  
    %Value contourline [mW/mm^2] (plots will be in mW/mm^2)
        v = [1,1];
    %Polynomial degree of contourfit
        g=3 ;
%------Initiations-------- 
    %Vector to store Fluence Rates
    dF=zeros(1000,1000,3);
    %Index for figures
    k=1; 
\end{lstlisting}
\textbf{MCML$\_$Data.m}
\begin{lstlisting}
function [R_Fig,z, Data_Fig, file_name] = MCML_Data(L,NA,path,protocol_name,l)
%MCML_Data.m Creates Fluence Rate for each grid point 
%   Reads the data file by using the function read_F
%   The data file is chosen by the protocol; protocol_name and the error
%   l=1 No error
%   l=2 +sigma u_a, +sigma u_s
%   l=3 -sigma u_a, -sigma u_s 
%   path is the loaction of the data files
%   Puts the data in the right configuration to use it in MCML_plot    

    if l==1
         file_name = ['L_' num2str(L) '_NA_' num2str(NA) ] ;
        file_Frz = [fullfile(path,'OutputMCML',[protocol_name '.Frzc'])] ;
    elseif l==2
         file_name = ['L_' num2str(L) '_NA_' num2str(NA) '+u_a+u_s']; 
        file_Frz = [fullfile(path,'OutputMCML+',[protocol_name '.Frzc'])];
    elseif l==3
         file_name = ['L_' num2str(L) '_NA_' num2str(NA) '-u_a-u_s']; 
        file_Frz = [fullfile(path,'OutputMCML-',[protocol_name '.Frzc'])];
    end
    [r,z,Data] = MCML_read_F(file_Frz);               %Read data, [mm,mm,/mm^2]                            
    R_Fig = [-r(end:-1:1), r];                                  
    Data_Fig = [Data(:,end:-1:1) Data]*1;        % * # mW/mm^2 @ tip
    Data_Fig= Data_Fig.*10^3; % SI units W/m^2                                        
end
\end{lstlisting}
\textbf{MCML$\_$read.m}
\begin{lstlisting}
function [r,z,Data] = MCML_read_F(MCML_file_MCO)
%read_F reads data from an .Frzc file
%   Reads in the data form teh MCML_file_MCO
%   changes the units
%   puts everything in the right format

    S= importdata(MCML_file_MCO);
    r = S.data(:,1); 
    z = S.data(:,2); 
    Data = S.data(:,3); 
    clear S;

    step_z = z(2)-z(1); z_min = min(z); z_max = max(z);
    
    z_length = round((z_max - z_min)./step_z)+1;
    r_length = length(r)./z_length;
    
    r = reshape(r,z_length,r_length); 
    r = r(1,:).*10;  % cm->mm
    z = reshape(z,z_length,r_length); 
    z = z(:,1)'.*10;% cm->mm
    Data = reshape(Data,z_length,r_length);
    Data(end,:) = 0;
    Data(:,end) = 0;
   Data = Data./100; %% /cm^2-> /mm^2

end

\end{lstlisting}
\subsubsection*{Temperatuursverandering continue stimulatie} % (fold)
\label{ssub:dtcontinue}
Onderstaande matlabcode leest de .csv bestanden uit de map \emph{unitFR} en berekent de temperatuursverandering voor een continue stimulatie met behulp van vergelijking \ref{eq:solBH}. Vervolgens plot hij op drie tijdstippen (\emph{Times$\_$to$\_$plot}) de temperatuursverandering. Tot slot worden nog twee grafieken (voor SM en MM) gemaakt van $\Delta T$ in functie van de tijd.\\
\textbf{Trise$\_$plot.m}
\begin{lstlisting}
Trise_Setup_Data
for i = 1: length(L);
    for j=1:length(NA)
        
        %--------Data--------------
        n_L=L(i); n_NA=NA(j);
        protocol_name = ['L_' num2str(n_L) '_NA_' num2str(n_NA) ];
        %define tissue properties
        prop(1)=u_a(i);    %absorption coefficient [m^-1]
        prop(2)=u_s(i) ;   %scattering coefficient [m^-1]
        prop(7)=w_L(j);   %1:e^ radius [m]
      
        %Read Fluence rate I=Unit peak irradiance  
        psi=csvread(fullfile(path,'UnitFR',protocol_name)); %[W/m^2]
        %Fluence rate +u_a
        psiplus=csvread(fullfile(path,'UnitFR',[protocol_name '+u_s']));
        %Fluence rate -u_a
        psiminus=csvread(fullfile(path,'UnitFR',[protocol_name '-u_s']));
        %Convert psi to Fluence rate phi for the asked Power/Irradiance
        if ne(PL,0) == 1
        IL=PL/((pi*prop(7)^2)/2);
        end
        %Save all data
        phi=IL.*psi;
        phiplus=IL.*psiplus; 
        phiminus=IL.*psiminus;
        
        %--------Calculation--------------
        if j==1
            for l=1:length(t1)
                
                %Calc. Temperature increade
                [dT(:,:,k,l), tau(k)]=Trise_Data(phi,0,0,prop,t1(l));
                
                [dTplus(k,l) p]=Trise_Data(phiplus(2,500),du_a_plus(i),...
                    du_s_plus(i),prop,t1(l));
                [dTminus(k,l), p]=Trise_Data(phiminus(2,500),du_a_minus(i),...
                    du_s_minus(i),prop,t1(l));
               
            end
        elseif j==2
            for l=1:length(t2)
                
                %calculate temperature raise
                [dT(:,:,k,l), tau(k)]=Trise_Data(phi,0,0,prop,t2(l));
                
                [dTplus(k,l),p]=Trise_Data(phiplus(2,500),du_a_plus(i),...
                   du_s_plus(i), prop,t2(l));
               [dTminus(k,l),p]=Trise_Data(phiminus(2,500),du_a_minus(i),...
                   du_s_minus(i),prop,t2(l));
               
                
            end
        end
        figure(m);
        imagesc(r,z,dT(:,:,k,timesToPlot(1))); c=colorbar;
        
        %scaling
        if j==1
            xlim([-0.5,0.5]);ylim([0,1]);
        elseif j==2
            xlim([-0.5,0.5]);ylim([0,1]);
        end
        if i==1 && j==1
            caxis([0 0.002]);
        elseif i==1 && j==2
            caxis([0 0.025]);
        elseif i==2 && j==1
            caxis([0 0.002]);
        elseif i==2 && j==2
            caxis([0 0.025]);
        end

        %labels
        if j==1
            title({'Temperature increase';['$\lambda=$' num2str(n_L)...
                ', NA=' num2str(n_NA) ', t=' ...
                num2str(t1(timesToPlot(1)))]},'interpreter', 'LaTex');
        elseif j==2
            title({'Temperature increase';['$\lambda=$' num2str(n_L)...
                ', NA=' num2str(n_NA) '; t='...
                num2str(t2(timesToPlot(1)))]},'interpreter', 'LaTex');
        end
        xlabel('r [mm]', 'interpreter', 'LaTex');
        ylabel('z [mm]', 'interpreter', 'LaTex');
        c.Label.String= '$dT$ [$K$]';
        c.Label.FontSize=20;
        c.Label.Interpreter='latex';
        

        %------------------------------------------------------------------    
        m=m+1  ;     
        figure(m);
        imagesc(r,z,dT(:,:,k,timesToPlot(2))); c=colorbar;
         %scaling
        if j==1
            xlim([-0.5,0.5]);ylim([0,1]);
        elseif j==2
            xlim([-0.5,0.5]);ylim([0,1]);
        end
        if i==1 && j==1
            caxis([0 0.002]);
        elseif i==1 && j==2
            caxis([0 0.025]);
        elseif i==2 && j==1
            caxis([0 0.002]);
        elseif i==2 && j==2
            caxis([0 0.025]);
        end
        %labels
        if j==1
            title({'Temperature increase';['$\lambda=$' num2str(n_L)...
                ', NA=' num2str(n_NA) ', t=' ...
                num2str(t1(timesToPlot(2)))]},'interpreter', 'LaTex');
        elseif j==2
            title({'Temperature increase';['$\lambda=$' num2str(n_L)...
                ', NA=' num2str(n_NA) '; t='...
                num2str(t2(timesToPlot(2)))]},'interpreter', 'LaTex');
        end
        xlabel('r [mm]', 'interpreter', 'LaTex');
        ylabel('z [mm]', 'interpreter', 'LaTex');
        c.Label.String= '$dT$ [$K$]';
        c.Label.FontSize=20;
        c.Label.Interpreter='latex';

        %------------------------------------------------------------------        
        m=m+1;       
        fig=figure(m);
        imagesc(r,z,dT(:,:,k,timesToPlot(3))); c=colorbar;
          %scaling
        if j==1
            xlim([-0.5,0.5]);ylim([0,1]);
        elseif j==2
            xlim([-0.5,0.5]);ylim([0,1]);
        end
        if i==1 && j==1
            caxis([0 0.002]);
        elseif i==1 && j==2
            caxis([0 0.025]);
        elseif i==2 && j==1
            caxis([0 0.002]);
        elseif i==2 && j==2
            caxis([0 0.025]);
        end

        %labels
        if j==1
            title({'Temperature increase';['$\lambda=$' num2str(n_L)...
                ', NA=' num2str(n_NA) ', t=' ...
                num2str(t1(timesToPlot(3)))]},'interpreter', 'LaTex');
        elseif j==2
            title({'Temperature increase';['$\lambda=$' num2str(n_L)...
                ', NA=' num2str(n_NA) '; t='...
                num2str(t2(timesToPlot(3)))]},'interpreter', 'LaTex');
        end
        xlabel('r [mm]', 'interpreter', 'LaTex');
        ylabel('z [mm]', 'interpreter', 'LaTex');
        c.Label.String= '$dT$ [$K$]';
        c.Label.FontSize=20;
        c.Label.Interpreter='latex';
        
        m=m+1;
        k=k+1;
        
        
    end
end
%------Plot--------
figure(m)
hold on
    plot(t1,squeeze(dT(2,500,1,:)),'Color',hex2rgb('352A86'));%blue474nm
    plot(t1,dTplus(1,:),'--','color',hex2rgb('352A86'))
    plot(t1,dTminus(1,:),'--','color',hex2rgb('352A86'))
    plot(tau(1),0,'r*');
            
    plot(t1,squeeze(dT(2,500,3,:)),'Color',hex2rgb('f1b94a'));%yellow 560nm
    plot(t1,dTplus(3,:),'--','color',hex2rgb('f1b94a'))
    plot(t1,dTminus(3,:),'--','color',hex2rgb('f1b94a'))
    plot(tau(3),0,'b*');
hold off
           
    xlabel('t [s]', 'interpreter', 'LaTex');
    ylabel('dT[K]', 'interpreter', 'LaTex');
m=m+1;
figure(m)
hold on
    plot(t2,squeeze(dT(2,500,2,:)),'Color',hex2rgb('352A86')); %blue 474nm
    plot(t2,dTplus(2,:),'--','color',hex2rgb('352A86'))
    plot(t2,dTminus(2,:),'--','color',hex2rgb('352A86'))
    plot(tau(2),0,'r*');
            
    plot(t2,squeeze(dT(2,500,4,:)),'Color',hex2rgb('f1b94a')); %yellow 560nm
    plot(t2,dTplus(4,:),'--','color',hex2rgb('f1b94a'))
    plot(t2,dTminus(4,:),'--','color',hex2rgb('f1b94a'))
    plot(tau(4),0,'b*');
hold off
           
    xlabel('t [s]', 'interpreter', 'LaTex');
    ylabel('dT[K]', 'interpreter', 'LaTex');          
\end{lstlisting}
\textbf{Trise$\_$Setup$\_$Data.m}
\begin{lstlisting}
%------Setup--------
close all
% Change default text font size
    set(0,'DefaultAxesFontSize', 18)
%File Path
    path=LD;
%------DATA--------

%Optical properties
    L = [474 560];          %Wavelengths(nm)
    NA = [12 37];           %Numerical Aperture (dimensionless)
    d = [9 200];            %Diameter (mum)
%Tissue properties
    u_a=[51.1 107.9];       % absorpion coefficient [m^-1]
    u_s=[12733 9266];       %scattering coefficient [m^-1]
    g=0.88;                 %anisotropy factor [unitless]
    p=1040;                 %tissue density [kg/m^3] 
    c=3650;                 %specific heat [J/(kg*K)]
    k_d=0.530;               %,thermal diffusivity [W/(m*K)]
    w_L= [0.0000045 0.0001];%1:e^ radius [m] 
%Errors
    du_a_plus=[34.9, 55.9];
    du_a_minus=[-34.9, -52.5];
    du_s_plus=[3267, 3558];
    du_s_minus=[-3267, -3492];
%Times to evaluate[s]       
    t1=linspace(0,0.000075,50);
    t2=linspace(0,0.040,50);
%Time indices to plot
    timesToPlot=[2 16 47];
%Laser Power [W] (default 0)
    PL=0.001;
%Irradiance [W/m^2] (default 1)
    IL=1;
 %positions in tissue
    filename_R=[fullfile(path,'FR_r')];
    filename_Z=[fullfile(path,'FR_z')];
    r=csvread(filename_R);  %radial position[mm]
    z=csvread(filename_Z);  %axial position[mm]
%-----Inistiations---
    %Tissue properties
    prop=[0 0 g p c k_d 0] ;%anisotropy factor;tissue density,
                            %specific heat,thermal diffusivity
    %Vector to store timeconstants
    tau=zeros(1,length(L)*length(NA)); 
    %Vector to store Temperature increases
    dT=zeros(1000,1000,4,length(t1)); 
    dTplus=zeros(4,length(t1));
    dTminus=zeros(4,length(t1));
    k=1%index for protocols
    m=1%index for figures    
\end{lstlisting}
\textbf{Trise$\_$Data.m}
\begin{lstlisting}
function [ dT, tau] = Trise_Data( phi,du_a,du_s,prop,t )
%Trise_Data Calculates the temperature increase 
    %This function uses the solution of the bioheat equation to calculate
    %the temperature increase. It use the fluence rate phi en the
    %properties in prop
    %prop(1)=u_a        absorpion coefficient [m^-1]
    %prop(2)=u_s        scattering coefficient [m^-1]
    %prop(3)=g          anisotropy factor [unitless]
    %prop(4)=p          tissue density [kg/m^3] 
    %prop(5)=c          specific heat [J/(kg*K)]
    %prop(6)=k          thermal diffusivity [W/(m*K)]
    %prop(7)=w_L        1/e^ radius [m] 
    %du_a ans du_s are the errors on the absorption and scattering
    %coefficient

prop(1)=du_a+prop(1);
prop(2)=du_s+prop(2);
r_0=prop( 7 ); 
z_0=2 / (prop(1) + ((1 - prop(3)) * prop(2))) ;
x=prop(4)*prop(5);
y=prop(6)*(2.4)^2;
z=r_0*pi/(4.8*z_0);
tau=(x/y)*(r_0^2/(1+z^2));

dT=(tau*prop(1)/(prop(4)*prop(5)))*(1-exp(-t/tau)).*phi;

end
\end{lstlisting}
\subsubsection*{Temperatuursverandering gepulste stimulatie} % (fold)
\label{ssub:dtcontinue}
Onderstaande matlabcode leest de .csv bestanden uit de map \emph{unitFR} en berekent de temperatuursverandering voor een gepulste stimulatie op de manier zoals uitgelegd in paragraaf \ref{sub:temperatuursverandering_simulatie}. Vervolgens wordt de temperatuursstijging op positie [(pos($1$),pos($2$)] geplot in functie van de tijd voor de vier stimulatie protocols. 
\textbf{Puls$\_$plot.m}
\begin{lstlisting}
%-------Setup-------
Puls_Setup_Data
%-------PLOT---------
for i = 1: length(L);
    for j=1:length(NA)
        %--------Data--------------
        n_L=L(i); n_NA=NA(j);
        protocol_name = ['L_' num2str(n_L) '_NA_' num2str(n_NA) ];
        %initialisation times to evaluate for each protocol
        t1=linspace(0,tL,ntss);
        t2=linspace(tL+t1(2),F-t1(2),ntsd); 
        %define tissue properties
        prop(1)=u_a(i);    %absorption coefficient [m^-1]
        prop(2)=u_s(i) ;   %scattering coefficient [m^-1]
        prop(7)=w_L(j);   %1:e^ radius [m]
        
        %Read Fluence rate I=Unit peak irradiance  
        psi=csvread(fullfile(path,protocol_name)); %[W/m^2]
        %Convert psi to Fluence rate phi for the asked Power/Irradiance
        if ne(PL,1) == 1 
        IL=PL/((pi*prop(7)^2)/2);
        end
        %Save all data
        phi=IL.*psi;
        %--------Calculation--------------
        %A loop over all the time windows
        for h= 1:t_w 
            if h==1 %first time window
                for l= 1:length(t1)
                    dT(1,l,k)=t1(l);
                    dT(2,l,k)=Rising(pos, prop, phi, t1(l) ,h);
                end
                for l=length(t1)+1: length(t1)+length(t2)
                    dT(1,l,k)=t2(l-length(t1));
                    dT(2,l,k)=Descending(pos, prop, phi, t2(l-length(t1)),h-1);
                end
            else %other time windows
                %first window is one rising and a sum over the previous
                %descending functions
                %The next one has only descending functions
                
                %rising
                for l= (h-1)*(length(t1)+length(t2))+1:...
                        (h-1)*(length(t1)+length(t2))+length(t1)
                    dT(1,l,k)=t1(l-((h-1)*(length(t1)+length(t2))));
                    dT(2,l,k)=Rising(pos,prop,phi,...
                        t1(l-(h-1)*(length(t1)+length(t2))),h); 
                %sum over previous descending functions
                    for p=1:h-1 
                        dT(2,l,k)=dT(2,l,k)+Descending(pos, prop, phi, ...
                            t1(l-(h-1)*(length(t1)+length(t2))),p-1);
                    end
                    
                end
                %sum over previous descending funcitons
                for l=(h-1)*(length(t1)+length(t2))+length(t1)+1:...
                        (h-1)*(length(t1)+length(t2))+length(t1)+length(t2)
                    dT(1,l,k)=t2(l-((h-1)*...
                        (length(t1)+length(t2))+length(t1)));
                    for p=1:h 
                        dT(2,l,k)=dT(2,l,k)+Descending(pos, prop, ...
                            phi,t2(l-((h-1)*...
                            (length(t1)+length(t2))+length(t1))),p-1);
                    end
                end
            end
            %Move the time one period
            t1=t1+F;
            t2=t2+F; 
        end

        k=k+1;
    end
    
end
fig=figure
set(fig, 'Position', [10 1000 1000 750]);
subplot(2,2,1)
plot(squeeze(dT(1,:,1)),squeeze(dT(2,:,1)))
xlabel('t [s]', 'interpreter', 'LaTex');
ylabel('dT [K]', 'interpreter', 'LaTex');
h=legend(['$\lambda=$' num2str(L(1))...
    ', NA=' num2str(NA(1))]);
set(h,'Interpreter','latex')
set(h,'Box','off')
grid on

subplot(2,2,3)
plot(squeeze(dT(1,:,2)),squeeze(dT(2,:,2)))
ylim([0 0.0010]);
xlabel('t [s]', 'interpreter', 'LaTex');
ylabel('dT [K]', 'interpreter', 'LaTex');
set(gca,'yaxislocation','right');
h=legend(['$\lambda=$' num2str(L(1))...
    ', NA=' num2str(NA(2))]);
set(h,'Interpreter','latex')
set(h,'Box','off')
grid on

subplot(2,2,2)
plot(squeeze(dT(1,:,3)),squeeze(dT(2,:,3)))

xlabel('t [s]', 'interpreter', 'LaTex');
ylabel('dT [K]', 'interpreter', 'LaTex');
h=legend(['$\lambda=$' num2str(L(2))...
    ', NA=' num2str(NA(1))]);
set(h,'Interpreter','latex')
set(h,'Box','off')
grid on
subplot(2,2,4)
plot(squeeze(dT(1,:,4)),squeeze(dT(2,:,4)))

xlabel('t [s]', 'interpreter', 'LaTex');
ylabel('dT [K]', 'interpreter', 'LaTex');
set(gca,'yaxislocation','right');
h=legend(['$\lambda=$' num2str(L(2))...
    ', NA=' num2str(NA(2))]);
set(h,'Interpreter','latex')
set(h,'Box','off')
grid on

shg;
p=mtit('Temperature increase for Pulsed stimulation for' ,...
    'fontsize',24,'interpreter', 'LaTex',...
    'xoff',0,'yoff',.05);

p=mtit( ['$\nu$=' num2str(freq) ' Hz , $t_L$=' num2str(tL) ' s , $I$=' num2str(IL) '$ \frac{W}{m^2}$'],...
    'fontsize',24,'interpreter', 'LaTex',...
    'xoff',0,'yoff',.015);
\end{lstlisting}
\textbf{Puls$\_$Setup$\_$Data.m}
\begin{lstlisting}
%------Setup--------
    close all
    % Change default text font size
        set(0,'DefaultAxesFontSize', 18)
        set(0,'defaulttextinterpreter','latex')
        
    %File Path
        LD =[fullfile('/Users','LIesDeceuninck', 'Documents','Bachelorproef',...
            'Data','Licht_verdeling','code','Data','UnitFR')];
        HV=[fullfile('C:','Users','hannelore','Documents','2015-2016',...
            'BachelorProef2015','BachelorProef','Data','Licht_verdeling','code','Data','UnitFR')];    
        path=LD;
%------DATA--------
%optical properties
    L = [474 560];          %Wavelengths(nm)
    NA = [12 37];           %Numerical Aperture (dimensionless)
    d = [9 200];            %Diameter (mum)
%tissue properties
    u_a=[51.1 107.9];       % absorpion coefficient [m^-1]
    u_s=[12733 9266];       %scattering coefficient [m^-1]
    g=0.88;                 %anisotropy factor [unitless]
    rho=1040;               %tissue density [kg/m^3] 
    c=3650;                 %specific heat [J/(kg*K)]
    k_d=0.530;              %,thermal diffusivity [W/(m*K)]
    w_L= [0.0000045 0.0001];%1:e^ radius [m] 
%Laser Power [W] (default 0)
    PL=1;
%Irradiance [W/m^2] (default 1)
    IL=46000
%Pulse length [s]
    tL=0.001; 
%Frequency of the pulses [s^-1]
    freq=40; 
    F=1/freq;               %time between pulses [s]
%Times to evaluate[s] 
    time=0.2;               %total time length
    ntss=400;               %number of time steps in a puls
    ntsd=500;               %number of time steps between pulses
    
    t1=linspace(0,tL,ntss); %times to evaluate in one puls
    t2=linspace(tL+t1(2),F-t1(2),ntsd); %times to evaluate between pulses
    
    t_w=floor(time/F);      %number of timewindows
%Position in the tissue to evaluate dT [z r]
    pos=[2 500];
%-----Initiations---
dT=zeros(2, (length(t1)+length(t2))*t_w , 4); %datacollection vector
prop=[0 0 g rho c k_d 0 tL F] ;         
                            %anisotropy factor;tissue density,specific 
                            %heat,thermal diffusivity,pulse length [s], 
                            %time between pulses[s]
k=1; %index for the protocols


\end{lstlisting}
\textbf{Rising.m}
\begin{lstlisting}
function [ dTs ] = Rising( pos,prop,phi,t,h)
%Rising Calculates the temperature increase during the pulse
    %This function uses the solution of the bioheat equation to calculate
    %the temperature increase in the tissue on positon [pos(1), pos(2)]. 
    %It use the fluence rate phi en the properties in prop
    %prop(1)=u_a        absorpion coefficient [m^-1]
    %prop(2)=u_s        scattering coefficient [m^-1]
    %prop(3)=g          anisotropy factor [unitless]
    %prop(4)=p          tissue density [kg/m^3] 
    %prop(5)=c          specific heat [J/(kg*K)]
    %prop(6)=k          thermal diffusivity [W/(m*K)]
    %prop(7)=w_L        1/e^ radius [m] 
    %du_a ans du_s are the errors on the absorption and scattering
    %coefficient
    %h is the time-window indes
r_0=prop(7); 
z_0=2 / (prop(1) + ((1 - prop(3)) * prop(2))) ;
x=prop(4)*prop(5);
y=prop(6)*(2.4)^2;
z=r_0*pi/(4.8*z_0);

tau=(x/y)*(r_0^2/(1+z^2));
dTss=((tau*prop(1))/x);

dTs=dTss*(1-exp(-(t-((h-1)*prop(9)) )/tau))*phi(pos(1),pos(2));
end
\end{lstlisting}
\textbf{Descending.m}
\begin{lstlisting}
function [ dTd ] = Descending( pos ,prop,phi,t,D)
%Descending Calculates the temperature decrease after a puls 
    %This function uses the time constant from the solution of the bioheat 
    %equation to calculate the temperature decrease. It's an descending 
    %exponential function with timeconstant tau. 
    %It use the fluence rate phi en the properties in prop
    %prop(1)=u_a        absorpion coefficient [m^-1]
    %prop(2)=u_s        scattering coefficient [m^-1]
    %prop(3)=g          anisotropy factor [unitless]
    %prop(4)=p          tissue density [kg/m^3] 
    %prop(5)=c          specific heat [J/(kg*K)]
    %prop(6)=k          thermal diffusivity [W/(m*K)]
    %prop(7)=w_L        1/e^ radius [m] 
    %du_a ans du_s are the errors on the absorption and scattering
    %coefficient
    %D is the time-window index

r_0=prop( 7 ); 
z_0=2 / (prop(1) + ((1 - prop(3)) * prop(2))) ;
x=prop(4)*prop(5);
y=prop(6)*(2.4)^2;
z=r_0*pi/(4.8*z_0);

tau=(x/y)*(r_0^2/(1+z^2));

dTss=((tau*prop(1))/x)*phi(pos(1),pos(2));
T0=dTss*(1-exp(-prop(8)/tau));

dTd=T0*exp(-(t-prop(8)-D*prop(9))/tau);
end
\end{lstlisting}
\newpage
\subsection*{Figuren} % (fold)
\label{sub:figuren}
\begin{figure*}[tb]
	\centering
	\includegraphics[width=\textwidth]{gfx/dT_Overig_bijlage.png}
	\caption{Temperatuurstijging voor continue stimulatie bij $474nm$, SM en MM en bij $560nm$, MM op 3 verschillende tijdstippen.}
	\label{fig:dT_Overig_Bijlage}
\end{figure*}

\begin{figure*}[tb]
	\centering
	\includegraphics[width=0.8\textwidth]{gfx/dT_time_mu-a_mu-sSM.png}
	\caption{De invloed van $\mu_a \pm \sigma$ en $\mu_s \pm \sigma$ op de temperatuurstijging voor SM.}
	\label{fig:foutSM}
\end{figure*}

\begin{figure*}[tb]
	\centering
	\includegraphics[width=\textwidth]{gfx/FR_scat+abs_variatie.png}
	\caption{De invloed van $\mu_a \pm \sigma$ en $\mu_s \pm \sigma$ op de lichtverdeling voor $474nm$, SM en MM en voor $560nm$, SM.}
	\label{fig:FR_scat+abs_variatie}
\end{figure*}
\twocolumn

