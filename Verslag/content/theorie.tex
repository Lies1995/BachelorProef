\section{Theoretische achtergrond} % (fold)

\label{sec:theoretische_achtergrond}
%#TODO eerste de eignelijke definitie tau, dan axiaal en radiaal
%psi en phi link uitleggen
%meer in detail op tau ingaan
%latex in orde krijgen
%pulsen uitleggen
\begin{figure}[h!]
	\centering
		\begin{subfigure}[b]{.5\textwidth}
			\centering
			\includegraphics[width=7cm]{gfx/exp.png}
			\caption{Temperatuurstijging bij continue laserstimulatie.}
			\label{fig:continue}
		\end{subfigure}%
		\\
		\begin{subfigure}[b]{.5\textwidth}
			\centering
			\includegraphics[width=7cm]{gfx/puls.png}
			\caption{Temperatuurstijging bij gepulste laserstimulatie.}
			\label{fig:puls}
		\end{subfigure}%
	\caption{}
	\label{fig:two}
\end{figure}

\subsection{De bioheat vergelijking}
De temperatuurstijging in hersenweefsel kan bepaald worden met behulp van de bioheat vergelijking:
\footnotesize
\begin{equation}
\label {eq:bioheat}
\begin{split}
\frac{\partial{\Delta T(t,z,r)}}{\partial{t}} = & \frac{\mu_{a}\phi(z,r)}{\rho c} +\\
 & \frac{k}{\rho c}\left[ \frac{\partial^2{\Delta 				T}}{\partial{z^2}}+\frac{\partial^2{\Delta T}}{\partial{r^2}}+\frac{1}{r}\frac{\partial{\Delta T}}{\partial{r}}\right].
\end{split}
%\frac{\partial{\Delta T(t,z,r)}}{\partial{t}} = \frac{\mu_{a}\phi(z,r)}{\rho c}+\frac{k}{\rho c}\left[ \frac{\partial^2{\Delta 				T}}{\partial{z^2}}+\frac{\partial^2{\Delta T}}{\partial{r^2}}+\frac{1}{r}\frac{\partial{\Delta T}}{\partial{r}}\right].
\end{equation}
\normalsize
Met $\Delta T$ de temperatuurstijging [$^{\circ}C$], $\mu_a$ de absorptieco\"effi\"ent van het weefsel [$m^{-1}$], $k$ de thermische geleidbaarheid [$Wm^{-1}^{\circ}C^{-1}$], $\rho$ de dichtheid van het weefsel [$kg$ $m^{-3}$], $c$ de warmtecapaciteit [$J$ $kg^{-1}^{\circ}C^{-1}$] en $\phi(z,r)$ de lichtverdeling van het laserlicht [$Wm^{-2}$]. Al deze parameters zijn gekend van voorgaande experimenten, behalve de lichtverdeling. Deze laatste zal berekend worden met behulp van een Monte Carlo methode, zie sectie \ref{sec:methode}. 

Het rechterlid van de bioheat vergelijking bestaat uit twee delen. Het eerste deel stelt de temperatuurstijging voor in het weefsel ten gevolge van absorptie van fotonen. Het tweede deel van het rechterlid toont de temperatuurstijging door warmtegeleiding. Aangezien deze laatste term negatief is, stelt ze dus een temperatuurdaling voor ten gevolge van warmtegeleiding.
\subsection{Oplossing van de bioheat vergelijking}
De bioheat vergelijking werd opgelost door Martin J.C. van Gemert met de methode van scheiding van variabelen. De oplossing wordt gegeven door: 
\begin{equation}
\Delta T(t,z,r) = \frac{\tau \mu_a\phi(z,r)}{\rho c}(1-e^{-\frac{t}{\tau}}).
\label{eq:solBH}
\end{equation}
Dit model zal verder gebruikt worden om de temperatuurstijging in hersenweefsel te bepalen.
Voor de oplossing werd een tijdsconstante $\tau$ ingevoerd, deze geeft een maat voor hoe snel de temperatuur in het weefsel zal stijgen ten gevolge van interactie met elektromagnetische straling. 
De tijdsconstante $\tau$ ziet er als volgt uit: 
\begin{equation}
\tau = \left( \frac{1}{\tau_z}+\frac{1}{\tau_r}\right)^{-1}
\end{equation}
Met $\tau_z$ de axiale en $\tau_r$ de radiale component van $\tau$. 
\begin{equation}
\frac{1}{\tau_z}=\frac{k}{\rho c}\left( \frac{\pi (\mu_a+(1-g)\mu_s)}{4}\right) ^{2}
\end{equation}
\begin{equation}
\frac{1}{\tau_r} =\frac{k}{\rho c} \left(\frac{2.4}{w_L}\right)^{2}
\end{equation}
Men kan zien dat de axiale component van $\tau$ bepaald wordt door weefselparameters $(\mu_a$, $g$ en $\mu_s)$ en de radiale component door de gebruikte optische vezel ($w_L$ is de straal van de vezel). Aangezien de weefselparameters $\mu_a$ en $\mu_s$ afhankelijk zijn van de golflengte, zal een verandering van golflengte enkel een invloed hebben op de axiale component van $\tau$. Anderzijds zal het gebruiken van een andere vezel enkel een invloed hebben op de radiale component van $\tau$. Dit laatste kan men eenvoudig intuitief verklaren. Een SM vezel is veel smaller dan een MM vezel ($w_L=0.00045cm$ t.o.v. $w_L=0.01cm$). Hierdoor komt bij een SM vezel alle energie op een smalle ruimte terecht, waardoor de warmte zich makkelijk in radiale richting kan verspreiden. Dit geeft een kleine $\tau_r$. Bij een MM vezel daarentegen is de energie over een veel groter gebied verspreidt, waardoor de warmte zich niet zo goed kan verspreiden in radiale richting. Dit leidt tot een grote $\tau_r$.   

Bij het opstellen van model (\ref{eq:solBH}) werden een aantal benaderingen gemaakt, zodat een analytische oplossing mogelijk was. Dit werd gedaan omdat een analytische oplossing meer inzicht geeft in het probleem dan een numerieke oplossing. 
De aannames die hiervoor gemaakt werden, staan weergegeven in tabel \ref{tab:aannames}. Deze zijn geldig voor optogenetische experimenten.  
De eerste aanname zegt dat de lichtverdeling evenredig moet zijn met de temperatuurstijging. Deze aanname is geldig omdat de lichtverdeling een vermogen per oppervlakte uitdrukt. In een gebied met een hoog vermogen is er evident ook een hoge temperatuurstijging. Omgekeerd zal in een gebied met een lager vermogen een lagere temperatuurstijging zijn.
We hebben nu dat 
\begin{equation}
\Delta T(t,z,r) = \theta(t)\phi(z,r)
\end{equation} 
Dit kan verder uitgewerkt worden met de methode van scheiding van variabelen met de juiste randvoorwaarden. De oplossingen zijn dan typisch $0^e$ orde Besselfuncties voor radiale temperaturen en cosinussen voor axiale temperaturen. Aangezien lichtverdeling en temperatuurstijging evenredig zijn, zijn dit ook oplossingen voor componenten van de lichtverdeling. De tweede aanname is met andere woorden ook geldig. 
Voor de derde aanname moet de temperatuurstijging en lichtverdeling $0$ geworden zijn voor $z>z_0=2/(\mu_a +(1-g)\mu_s)$ en $r>r_0=w_L$. Aan deze voorwaarde is voldaan wanneer de thermische diffusielengte kleiner is dan $r_0$ en $z_0$. De thermische diffusielengte is gelijk aan $\sqrt{4\frac{k}{\rho c}t_{puls}}$. Voor $z_0$ levert dit geen problemen op. Voor $r_0$ echter betekend dit een sterke limitering op de pulstengte. Met behulp van tabel \ref{tab:thermEig} kan hieruit de maximale pulslengte berekend worden. Voor de single-mode is dat $2.6\cdot10^{-5}s$ en voor multi-mode $1.7\cdot10^{-2}s$. Aan de derde voorwaarde zal hierdoor niet altijd voldaan zijn.
Tot slot zijn de optische en thermische eigenschappen van het weefsel temperatuursafhankelijk, maar de temperatuurstijgingen bij optogenetische experimenten zijn te klein op een significante verandering te veroorzaken. Aan de laatste voorwaarde zal bijgevolg voldaan zijn. 

\begin{table}[!ht]
	\caption{Gemaakte aannames om tot een oplossing van de bioheat vergelijking te komen.}
	\label{tab:aannames}
	\centering
	\begin{tabular}{l p{7cm}}
	\hline
	\hline
		$1.$ & De lichtverdeling is op elk tijdstip evenredig met de ruimtelijke verdeling van de temperatuurstijging. \\
		$2.$ & De radiale verdeling van de lichtverdeling kan voorgesteld worden door een $0^e$ orde Besselfunctie en de axiale verdeling door een cosinus.\\
		$3.$ & De temperatuurstijging en lichtverdeling zijn $0$ geworden voor $z>z_0$ en $r>r_0$.\\
		$4.$ & De optische en thermische eigenschappen van het weefsel zijn onafhankelijk van de temperatuur.\\
	\hline
	\hline
	\end{tabular}
\end{table}
Deze benaderingen hebben wel tot gevolg dat de oplossing niet meer exact is. In figuur \ref{fig:comparison} wordt de oplossing gevonden door tijdsconstante analyse (\ref{eq:solBH}) vergeleken met een exacte numerieke oplossing en een oplossing die uitgaat van geen warmtegeleiding. Men ziet dat de temperatuur voor de oplossing zonder warmtegeleiding naar oneindig gaat, geleiding is immers de enige factor die de temperatuur kan laten afnemen. Geen warmtegeleiding zorgt er dus voor dat de temperatuur onbeperkt kan toenemen. De exacte oplossing en die met tijdsconstanten zullen satureren, weliswaar op een andere temperatuur. Op de figuur is te zien dat de oplossing met tijdsconstanten slecht voor een korte tijdsduur overeenkomt met de exacte numerieke oplossing. Vanaf een tijdsduur van langer dan ongeveer drie keer de tijdsconstante, wijkt oplossing (\ref{eq:solBH}) te sterk af van de exacte oplossing. De rede hiervoor is dat $\tau$ eigenlijk temperatuur afhankelijk is. In het begin van de stimulatie zit alle energie in een klein gebied. Hierdoor kan de warmte zich snel verspreiden, waardoor $\tau$ klein is. Na langere stimulatie is de energie meer verspreid, waardoor de warmte minder goed weg kan. Hierdoor zal $\tau$ groter worden. In het gebruikte model wordt $\tau$ als een constante gezien en wordt dus geen rekening gehouden met de tijdsafhankelijkheid. Bovendien zal voorwaarde 3 uit tabel \ref{tab:aannames} niet altijd voldaan zijn. Hierdoor zal de benaderende oplossing slechts bruikbaar zijn voor laserstimulaties korter dan $3\tau$.

\begin{figure}[tb]
  	\centering
  	\includegraphics[width=8.7cm]{gfx/comparison.png}
  	\caption{Vergelijking van de drie modellen. Een model dat warmtegeleiding verwaarloost, de exacte numerieke oplossing en het model dat gebruik maakt van tijdsconstanten. }
  	\label{fig:comparison}
  \end{figure} 
			
Het model met de tijdsconstanten (\ref{eq:solBH}) kan nu gebruikt worden om de temperatuurstijging te meten in hersenweefsel door continue en gepulste lasterstimulatie.
 
\subsection{Continue stimulatie}
Voor de continue stimulatie zal de temperatuur in het begin snel stijgt en dat er na een bepaalde tijd saturatie optreedt, zoals te zien is op figuur \ref{fig:continue}. Dit komt omdat er enerzijds opwarming zal zijn door absorptie, maar anderzijds ook afkoeling door warmtegeleiding. In het begin zal de absorptie groter zijn dan de geleiding, waardoor de temperatuur sterk stijgt. Na een tijd echter zal de geleiding steeds groter worden, tot ze even groot is als de absorptie en er saturatie optreedt. 

\subsection{Gepulste stimulatie}
Om de fysiologische omstandigheden in de neuronen zo goed mogelijk te benaderen, wordt vaak gebruik gemaakt van gepulste stimulatie. Tijdens een puls verwacht men, zoals bij de continue stimulatie, een exponentiele stijging van de temperatuur. Tussen twee pulsen zal de temperatuur exponentiel dalen, met dezelfde tijdsconstante als dat de temperatuur steeg. Bij de volgende puls zal de temperatuur opnieuw stijgen. Er moet echter rekening gehouden worden met het feit dat de temperatuurstijging ten gevolge van de vorige puls nog niet helemaal naar nul gezakt is, deze zal er dus bij opgeteld moeten worden. Bij de puls die daarop volgt, moet dan de temperatuurstijging van de twee vorige pulsen meegeteld worden. De temperatuurstijging tijdens een puls kan dus telkens geschreven worden als een som van een stijgende functie en allemaal dalende functies ten gevolge van de voorgaande pulsen. De temperatuurstijging tussen twee pulsen is dan een som van enkel dalende functies. Op figuur \ref{fig:puls4} staan de pulsen getekend met telkens de bijhorende temperatuurstijging. Op figuur \ref{fig:puls} kan men een typische vorm zien, die men krijgt wanneer de contributies van de verschillende pulsen worden opgeteld.

\begin{figure}[tb]
  	\centering
  	\includegraphics[width=8.7cm]{gfx/puls4.png}
  	\caption{Pulsen met bijhordende temperatuurstijging. }
  	\label{fig:puls4}
  \end{figure} 

Bij de gepulste stimulatie kan er op twee manieren saturatie optreden. Enerzijds zal er, wanneer een puls lang genoeg duurt, saturatie optreden binnen in een puls, net zoals bij de continue stimulatie. Anderzijds zal na een bepaalde tijd de bijdrage van de eerste puls naar nul gezakt zijn. Bij de daarop volgende puls zal ook de bijdrage van de tweede puls niet meer relevant zijn. Vanaf dat moment zal er dus telkens een gelijk aantal dalende functies zijn die een niet-triviale bijdrage leveren, waardoor de daaropvolgende pieken telkens even hoog zullen zijn. Dit is duidelijk te zien op figuur \ref{fig:puls}. Men kan zien dat de vierde en de vijfde piek even hoog zijn, dit komt omdat de bijdrage van de eerste puls bij de vierde piek reeds naar nul gezakt is. 




