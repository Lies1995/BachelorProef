\section{Conclusie}
De temperatuurstijging ten gevolgde van laser of led stimulatie werd berekend voor 4 protocols, namelijk $474nm$/$560nm$ en SM/MM. Dit werd gedaan voor zowel continue stimulatie als voor gepulste stimulatie. Voor deze laatste werden de berekeningen gedaan voor verschillende irradianties, pulslengtes en frequenties.
Door de beperkte geldigheid van het model, kon het niet gebruikt worden voor single-mode vezels bij gepulste stimulatie, daar de pulslengte telkens veel langer was dan $3\tau$. Om de temperatuurstijging hiervoor te bepalen zal men dus een ander model moeten aanwenden. Een mogelijkheid is om een exacte numerieke oplossing van de bioheat vergelijking te gebruiken, dit levert echter minder inzicht in de relatie tussen de verschillende weefsel en laser of led parameters en de temperatuursverandering. 
Het gebruikte model voorspelt geen weefselschade ten gevolge van temperatuurstijging, aangezien deze ruim onder $0.5^{\circ}C$ blijft. Voor de gepulste stimulatie werden geen fouten berekend, hier is het dus eventueel mogelijk dat met de foutenintervals erbij, de temperatuur toch te hoog kan worden. Hierbij moet zeker opgemerkt worden dat bovenstaande redenering enkel geld binnen het gebruikte model. Dit model was slechts een benadering, die bovendien een lagere temperatuurstijging uitkomt dan een exacte numerieke oplossing. Hierdoor is het mogelijk dat, ook al geeft het gebruikte model geen te hoge temperatuurstijging, de werkelijke temperatuurstijging hoger ligt en mogelijk dus weefselschade kan veroorzaken. Voor een betere benadering van de werkelijke temperatuurstijging zullen numerieke methodes gebruikt moeten worden. 
Zoals eerder al werd vermeld, werd er geen foutenanalyse gedaan bij de gepulste stimulatie. Dit is echter zeer interessant om in toekomstige onderzoeken wel te doen.    