\documentclass[paper=a4, fontsize=11pt]{scrartcl} % A4 paper and 11pt font size

\usepackage[T1]{fontenc} % Use 8-bit encoding that has 256 glyphs
\usepackage{fourier} % Use the Adobe Utopia font for the document - comment this line to return to the LaTeX default
\usepackage[english]{babel} % English language/hyphenation
\usepackage{amsmath,amsfonts,amsthm} % Math packages
\usepackage{xcolor}
\usepackage{url}
\usepackage{array}

\usepackage{sectsty} % Allows customizing section commands
\allsectionsfont{\centering \normalfont\scshape} % Make all sections centered, the default font and small caps

\usepackage{fancyhdr} % Custom headers and footers
\pagestyle{fancyplain} % Makes all pages in the document conform to the custom headers and footers
\fancyhead[R]{Lies Deceuninck, Hannelore Verhoeven} % No page header - if you want one, create it in the same way as the footers below
\fancyfoot[L]{Bachelor Thesis(1) 2015} % Empty left footerœ
\fancyfoot[C]{KuLeuven} % Empty center footer
\fancyfoot[R]{\thepage} % Page numbering for right footer
\renewcommand{\headrulewidth}{0pt} % Remove header underlines
\renewcommand{\footrulewidth}{0pt} % Remove footer underlines
\setlength{\headheight}{13.6pt} % Customize the height of the header

\numberwithin{equation}{section} % Number equations within sections (i.e. 1.1, 1.2, 2.1, 2.2 instead of 1, 2, 3, 4)
\numberwithin{figure}{section} % Number figures within sections (i.e. 1.1, 1.2, 2.1, 2.2 instead of 1, 2, 3, 4)
\numberwithin{table}{section} % Number tables within sections (i.e. 1.1, 1.2, 2.1, 2.2 instead of 1, 2, 3, 4)

\setlength\parindent{0pt} % Removes all indentation from paragraphs - comment this line for an assignment with lots of text
%----------------------------------------------------------------------------------------
%	TITLE SECTION
%----------------------------------------------------------------------------------------

\newcommand{\horrule}[1]{\rule{\linewidth}{#1}} % Create horizontal rule command with 1 argument of height

\title{	
\normalfont \normalsize 
\textsc{Kuleuven, Department of Physics and Astronomy} \\ [25pt] % Your university, school and/or department name(s)
\horrule{0.5pt} \\[0.2cm] % Thin top horizontal rule
\huge Bachelor Thesis\\ % The assignment title
\horrule{2pt} \\[0.3cm] % Thick bottom horizontal rule
}

\author{Lies Deceuninck and Hannelore Verhoeven} % Your name

\date{\normalsize\today} % Today's date or a custom date

\begin{document}

\maketitle % Print the title

\section{Emails} % (fold)
\label{sec:emails}
\subsection{Nerf} % (fold)
\label{sub:nerf}


Hi Davide\\
How are you? (I hope you still remember me, physics student with an internship at your lab two years ago. We met again I think last year?
Yes, I$'$m sill very sure about becoming a researcher in neuroscience. I finished the second year of physics successfully; another step closer to what I want.\\
Third year of Bachelor means Bachelor thesis and it is in groups of two. I convinced my colleague (Hannelore Verhoeven) to pick the topic about neuroscience \emph{Light distribution and thermal effects in the rat brain under optogenetic stimulation} .\\
Our host is Barbara Gysbrechts. Perhaps you know her?  I saw on the website of Nerf that she participates/participated in the lab of Francesco Battagalia.
Hannelore and I are now both reading publications and papers about optogenetics to get to know the field. Our work will be purely theoretical though; modelling the properties of light in different tissues (on the computer).\\
So here$'$s my question for you. Do you think it would be possible to show me and especially Hannelore around in the lab to see some of te practical aspects of neuroscience? It helps to get the whole picture of the experiment. 
But more importantly, concerning our bachelor thesis, do you think somebody from the lab, who$'$s working with optogenetics, would be willing to show us some of those experiments and explain his techniques?\\
And if it$'$s possible, could that happen next week or the week after? I know it$'$s soon, but we got our topic yesterday and our first evaluation is (already) 4 november.\\
Thanks a lot and see you soon (hopefully)!\\
Lies Deceuninck\\
\newpage
\textcolor{gray}{
Hello Lies\\
Of course I remember you. Nice to hear from you! I'm glad your interest for neuroscience hasn$'$t subsided and that you are spreading your interest to your co-students. If you bring Hannelore to Nerf, I can show her the lab and explain a bit of what we do.\\
I saw Barbara around the lab, but I don$'$t know her personally. Anyway you chose a very interesting topic: congrats!\\
At the moment we have only one person actively working with optogenetics in our lab. That$'$s JJ, our post-doctoral researcher (in CC). Unfortunately, he$'$s not going to perform optogenetics experiments until the 2nd half of November, but he can answer some of your questions regarding the practical use of this technique in freely behaving animals, together with the scientific aspects of its use (which questions about brain physiology he's addressing with it).\\
Hope to see you soon too, and let me know if I can be of any other help.\\
Cheers,\\
Davide\\
}
\end{document}

