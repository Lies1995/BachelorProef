\documentclass[paper=a4, fontsize=11pt]{scrartcl} % A4 paper and 11pt font size

\usepackage[T1]{fontenc} % Use 8-bit encoding that has 256 glyphs
\usepackage{fourier} 
\usepackage{graphicx}
\graphicspath{{gfx/}}
\usepackage[english]{babel} % English language/hyphenation
\usepackage{amsmath,amsfonts,amsthm} % Math packages
\usepackage{xcolor}
\usepackage{url}
\usepackage{array}
\usepackage{marginnote}
\usepackage[top=2cm, bottom=2cm, outer=5cm, inner=2cm, heightrounded, marginparwidth=3.5cm, marginparsep=0.5cm]{geometry}

\usepackage{sectsty} % Allows customizing section commands
\allsectionsfont{\centering \normalfont\scshape} % Make all sections centered, the default font and small caps

\usepackage{fancyhdr} % Custom headers and footers
\pagestyle{fancyplain} % Makes all pages in the document conform to the custom headers and footers
\fancyhead[R]{Lies Deceuninck, Hannelore Verhoeven} % No page header - if you want one, create it in the same way as the footers below
\fancyfoot[L]{Bachelor Thesis(1) 2015} % Empty left footerœ
\fancyfoot[C]{KuLeuven} % Empty center footer
\fancyfoot[R]{\thepage} % Page numbering for right footer
\renewcommand{\headrulewidth}{0pt} % Remove header underlines
\renewcommand{\footrulewidth}{0pt} % Remove footer underlines
\setlength{\headheight}{13.6pt} % Customize the height of the header

\numberwithin{equation}{section} % Number equations within sections (i.e. 1.1, 1.2, 2.1, 2.2 instead of 1, 2, 3, 4)
\numberwithin{figure}{section} % Number figures within sections (i.e. 1.1, 1.2, 2.1, 2.2 instead of 1, 2, 3, 4)
\numberwithin{table}{section} % Number tables within sections (i.e. 1.1, 1.2, 2.1, 2.2 instead of 1, 2, 3, 4)

\setlength\parindent{0pt} % Removes all indentation from paragraphs - comment this line for an assignment with lots of text
%----------------------------------------------------------------------------------------
%	TITLE SECTION
%----------------------------------------------------------------------------------------

\newcommand{\horrule}[1]{\rule{\linewidth}{#1}} % Create horizontal rule command with 1 argument of height

\title{	
\normalfont \normalsize 
\textsc{Kuleuven, Department of Physics and Astronomy} \\ [25pt] % Your university, school and/or department name(s)
\horrule{0.5pt} \\[0.2cm] % Thin top horizontal rule
\huge Bachelor Thesis\\ % The assignment title
\horrule{2pt} \\[0.3cm] % Thick bottom horizontal rule
}

\author{Lies Deceuninck and Hannelore Verhoeven} % Your name

\date{\normalsize\today} % Today's date or a custom date

\begin{document}

\maketitle % Print the title

\section{Emails} % (fold)
\label{sec:emails}
\subsection{Nerf} % (fold)
\label{sub:nerf}


Hi Davide\\
How are you? (I hope you still remember me, physics student with an internship at your lab two years ago. We met again I think last year?
Yes, I$'$m sill very sure about becoming a researcher in neuroscience. I finished the second year of physics successfully; another step closer to what I want.\\
Third year of Bachelor means Bachelor thesis and it is in groups of two. I convinced my colleague (Hannelore Verhoeven) to pick the topic about neuroscience \emph{Light distribution and thermal effects in the rat brain under optogenetic stimulation} .\\
Our host is Barbara Gysbrechts. Perhaps you know her?  I saw on the website of Nerf that she participates/participated in the lab of Francesco Battagalia.
Hannelore and I are now both reading publications and papers about optogenetics to get to know the field. Our work will be purely theoretical though; modelling the properties of light in different tissues (on the computer).\\
So here$'$s my question for you. Do you think it would be possible to show me and especially Hannelore around in the lab to see some of te practical aspects of neuroscience? It helps to get the whole picture of the experiment. 
But more importantly, concerning our bachelor thesis, do you think somebody from the lab, who$'$s working with optogenetics, would be willing to show us some of those experiments and explain his techniques?\\
And if it$'$s possible, could that happen next week or the week after? I know it$'$s soon, but we got our topic yesterday and our first evaluation is (already) 4 november.\\
Thanks a lot and see you soon (hopefully)!\\
Lies Deceuninck\\
\newpage
\textcolor{gray}{
Hello Lies\\
Of course I remember you. Nice to hear from you! I'm glad your interest for neuroscience hasn$'$t subsided and that you are spreading your interest to your co-students. If you bring Hannelore to Nerf, I can show her the lab and explain a bit of what we do.\\
I saw Barbara around the lab, but I don$'$t know her personally. Anyway you chose a very interesting topic: congrats!\\
At the moment we have only one person actively working with optogenetics in our lab. That$'$s JJ, our post-doctoral researcher (in CC). Unfortunately, he$'$s not going to perform optogenetics experiments until the 2nd half of November, but he can answer some of your questions regarding the practical use of this technique in freely behaving animals, together with the scientific aspects of its use (which questions about brain physiology he's addressing with it).\\
Hope to see you soon too, and let me know if I can be of any other help.\\
Cheers,\\
Davide\\}
\\
Davide \\
Hannelore and I are alway available on Wednesday and Friday\\
Perhaps you could introduce us to the lab and your techniques first and JJ  could continue in the afternoon?\\
Let me know what suits you best.\\
Thank you for doing this!\\
Lies \\

\textcolor{gray}{
Hello Lies\\
Shall we do it on 21st October then? \\
JJ will be available in the morning, as in the afternoon he's busy with experiments. You can come in the middle morning,  we talk a bit, I'll show you the lab, you talk with him and then we have lunch together. \\If there's chance (but I don't promise as it doesn't depend on me) in the afternoon you can attend some of the experiments going on in the lab.\\
Let me know.\\
Cheers,\\
Davide
}
% 
\section{Notes} % (fold)
\label{sec:notes}
\subsection{On Light distribution and thermal effects in the rat brain under optogenetic stimulation} % (fold)
\label{sub:article_n_1}
\emph{Optogenetics refers to the integration of optics and genetics to achieve gain- or loss-of-function of well-defined events within specific cells of living tissue}\cite{OptoGen}
\\
There are different ways to record and studie brainsignals. Electrical with electrodes and with photons-->Optogenetics (\emph{Optical brain stimulation}. That is what we will do. \\
Different organisms have beautifully evolved to obtaining mechanisms that can harvest light, using it to maintain their membrane potentials or identify suitable life environments. \cite{Opsin}
An important class of these light-sensitive proteins (microbial opsins) are transmembrane \emph{rhodopsins}. These proteins are useful
\begin{itemize}
	\item activity control by light
	\item encoded by only one gene
	\item fast kinetics
\end{itemize}
Mostly they are ion channels used to regulate the membrane potential. 
Over the years, many research has been done to genetically modify mammalian neurons to have these light-sensitive proteins expressed in the cell. Consequently, we have the ability to control cell activity of neurons and record action potentials in the brains of mammals.
\\
The Opsins are also modified to have a specific funcitonallity and to be sesitive for a specific wavelength. 


why photogenetics?
\begin{itemize}
 	\item superior cell-type specificity
 	\item no electrical artifact
 	\item minimized tissue damage

 \end{itemize} 
 In order to do experiment correctly and safely, we need an stimulation protocol; \emph{What is the ideal intensity to optimize stimulation and minimize damage?}
\marginnote{-->the propagation of light in the different areas of the brain; what kind of ifluence has light on the neurons.\cite{First} }\\
To target a region in the brain, they use optical fibers (multi-mode (100 to 200 $\mu$m) or single mode + a fiber for stimulation and recording at once+ multiple brain regions). The goal is to minimize the illumination of braintissue--> High spatial resolution.
The influence on te brain depends on the light intensity profile in the tissue. This profile depends on several things.\\
\marginnote{refractive index : n=$\frac{c}{v}$, Snell's law: $n_1\text{sin}\theta_1= n_2\text{sin}\theta_2$\\ Scattering coefficient: how much attenuation occurs by scattering}
\emph{Variables for the lightsource}
\begin{itemize}
	\item wavelenght
	\item sourcePower
	\item $NA$ (numerical aperture)
	\item core diameter $d_{core}$
	\item irradiation time
	\item repetition rate
\end{itemize}
\emph{Variables for the braintissue (Wavelength dependent)}\\
\marginnote{absorption $<$ Scattering in tissue. a)Rayleigh scattering--> isotropic, b) Mie regime--> anisotropic--> in cells. g determine how anisotropic.}
Optical properties
\begin{itemize}
	\item absorption coefficient $\mu_{a}$
	\item Scattering coefficient $\mu_{s}$
	\item refractive index $n$
	\item anisotropy factor $g$
\end{itemize}
Thermal properties
\begin{itemize}
	\item density
	\item Heat capacity
	\item thermal conductivity
\end{itemize}
\begin{figure}[!h]
	\centering
	\includegraphics[width=12cm]{scattering}
	\caption{Different scatterings}
	\label{fig:Scattering}
\end{figure}
In this studie, optical coefficients are measured in fuction of these properties for different rat brain tissues. \emph{Contact spatially resolved spectroscopy} was used to study the brains. The different investigated brain regions are:
\begin{itemize}
	\item cortex (CO) 
	\item hippocampus (HC) 
	\item striatuma (ST)
	\item thalamus (TH)
\end{itemize}
\begin{figure}[!h]
	\centering
	\includegraphics[]{RatBrain}
	\caption{Schematic overvieuw of the Rat Brain}
	\label{fig:RatBrain}
\end{figure}
In all mesurements, the treshold for Opsin stimulation in $1mW/mm^2$. \marginnote{Measurements in each brain section
\begin{itemize}
	\item preformed within $1$ minute
	\item repeted $5$ times (minimize noize)
	\item CO: $10$ different brains
	\item HC, ST,TH: $7$ different brains 
	\item dark room
	\item within $4h$ after extraction
\end{itemize}
}
\section{Vocabulairy} % (fold)
\label{sec:vocabulary}
This section contains a list of domain specific vocabulairy \cite{Youtube} \\
\begin{tabular}{p{2cm}|p{5cm}|p{6cm}}
Attenuation & The gradual loss in intensity of any kind of fluc through a medium & Attenuation can be caused by the quality of the glass in the optical fiber or by micro/macro bending \\ \hline
Optical fiber & flexible thin fiber used for data transmittance & An Optical fiber consists of a core and cladding both existing out of glass.It works on the principle of total internal reflection.\\	\hline
Single mode fiber & Optical fiber with only one lightsignal & Thin core that allows only one light beam\\ \hline
Multimode fiber & Optical fiber with more ligthsignals at once & Bigger core \\ \hline
Cutoff wavelength & the wavelength above which the wire will only support single mode signals & propertie of a single mode optical fibers \\ \hline
Mode field diameter & the diameter of the optical distribution in the fiber & propertie of a single mode fiber,some of the light (almost 30 $\%$) propagates in the cladding layer. This is a measuere of how much propagation happens\\ \hline
Numerical aperture & the measure of the angular range in which the fiber accepts light & it depends on the refractive indices of the core and cladding glass \\ \hline
Core size & design propertie of the fiber & the larger the core, the more moving light can propagate throug the fiber (more modes)\\ \hline
Acquisition time & time necessary to get the data of a measurement ($140ms$)& the time between two measurements here was 1s\\ \hline

\end{tabular}

% section vocabulary (end)
\bibliographystyle{plain}
\bibliography{Bibliography}



\end{document}













